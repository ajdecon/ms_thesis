%%
%% This is file `thesis-ex.tex',
%% generated with the docstrip utility.
%%
%% The original source files were:
%%
%% uiucthesis2009.dtx  (with options: `example')
%% 
\def\fileversion{v2.25a} \def\filedate{2009/10/10}
%% Package and Class "uiucthesis2009" for use with LaTeX2e.
\documentclass[edeposit,fullpage]{uiucthesis2009}

\begin{document}

\title{Coffee Consumption of Graduate Students \\
       Trying to Finish Dissertations}
\author{Juan Valdez}
\department{Food Science}
\schools{B.A., University of Columbia, 1981\\
         A.M., University of Illinois at Urbana-Champaign, 1986}
\phdthesis
\advisor{Java Jack}
\degreeyear{1994}
\committee{Professor Prof Uno, Chair\\Professor Prof Dos, Director of Research\\Assistant Professor Prof Tres\\Adjunct Professor Prof Quatro}
\maketitle

\frontmatter

%% Create an abstract that can also be used for the ProQuest abstract.
%% Note that ProQuest truncates their abstracts at 350 words.
\begin{abstract}
This is a comprehensive study of caffeine consumption by graduate
students at the University of Illinois who are in the very final
stages of completing their doctoral degrees. A study group of six
hundred doctoral students\ldots.
\end{abstract}

%% Create a dedication in italics with no heading, centered vertically
%% on the page.
\begin{dedication}
To Father and Mother.
\end{dedication}

%% Create an Acknowledgements page, many departments require you to
%% include funding support in this.
\chapter*{Acknowledgments}

This project would not have been possible without the support of
many people. Many thanks to my adviser, Lawrence T. Strongarm, who
read my numerous revisions and helped make some sense of the
confusion. Also thanks to my committee members, Reginald Bottoms,
Karin Vegas, and Cindy Willy, who offered guidance and support.
Thanks to the University of Illinois Graduate College for awarding
me a Dissertation Completion Fellowship, providing me with the
financial means to complete this project. And finally, thanks to
my husband, parents, and numerous friends who endured this long
process with me, always offering support and love.

%% The thesis format requires the Table of Contents to come
%% before any other major sections, all of these sections after
%% the Table of Contents must be listed therein (i.e., use \chapter,
%% not \chapter*).  Common sections to have between the Table of
%% Contents and the main text are:
%%
%% List of Tables
%% List of Figures
%% List Symbols and/or Abbreviations
%% etc.

\tableofcontents
\listoftables
\listoffigures

%% Create a List of Abbreviations. The left column
%% is 1 inch wide and left-justified
\chapter{List of Abbreviations}

\begin{symbollist*}
\item[CA] Caffeine Addict.
\item[CD] Coffee Drinker.
\end{symbollist*}

%% Create a List of Symbols. The left column
%% is 0.7 inch wide and centered
\chapter{List of Symbols}

\begin{symbollist}[0.7in]
\item[$\tau$] Time taken to drink one cup of coffee.
\item[$\mu$g] Micrograms (of caffeine, generally).
\end{symbollist}

\mainmatter
\chapter{This world}
\section{Of the Nature of Flatland}

I call our world Flatland, not because we call it so, but to make its
nature clearer to you, my happy readers, who are privileged to live in
Space.

Imagine a vast sheet of paper on which straight Lines, Triangles,
Squares, Pentagons, Hexagons, and other figures, instead of remaining
fixed in their places, move freely about, on or in the surface, but
without the power of rising above or sinking below it, very much like
shadows--only hard with luminous edges--and you will then have a pretty
correct notion of my country and countrymen.  Alas, a few years ago, I
should have said "my universe:"  but now my mind has been opened to
higher views of things.

In such a country, you will perceive at once that it is impossible that
there should be anything of what you call a "solid" kind; but I dare
say you will suppose that we could at least distinguish by sight the
Triangles, Squares, and other figures, moving about as I have described
them.  On the contrary, we could see nothing of the kind, not at least
so as to distinguish one figure from another.  Nothing was visible, nor
could be visible, to us, except Straight Lines; and the necessity of
this I will speedily demonstrate.

Place a penny on the middle of one of your tables in Space; and leaning
over it, look down upon it.  It will appear a circle.

But now, drawing back to the edge of the table, gradually lower your
eye (thus bringing yourself more and more into the condition of the
inhabitants of Flatland), and you will find the penny becoming more and
more oval to your view, and at last when you have placed your eye
exactly on the edge of the table (so that you are, as it were, actually
a Flatlander) the penny will then have ceased to appear oval at all,
and will have become, so far as you can see, a straight line.

The same thing would happen if you were to treat in the same way a
Triangle, or a Square, or any other figure cut out from pasteboard.  As
soon as you look at it with your eye on the edge of the table, you will
find that it ceases to appear to you as a figure, and that it becomes
in appearance a straight line.  Take for example an equilateral
Triangle--who represents with us a Tradesman of the respectable class.
Figure 1 represents the Tradesman as you would see him while you were
bending over him from above; figures 2 and 3 represent the Tradesman,
as you would see him if your eye were close to the level, or all but on
the level of the table; and if your eye were quite on the level of the
table (and that is how we see him in Flatland) you would see nothing
but a straight line.

When I was in Spaceland I heard that your sailors have very similar
experiences while they traverse your seas and discern some distant
island or coast lying on the horizon.  The far-off land may have bays,
forelands, angles in and out to any number and extent; yet at a
distance you see none of these (unless indeed your sun shines bright
upon them revealing the projections and retirements by means of light
and shade), nothing but a grey unbroken line upon the water.

Well, that is just what we see when one of our triangular or other
acquaintances comes towards us in Flatland.  As there is neither sun
with us, nor any light of such a kind as to make shadows, we have none
of the helps to the sight that you have in Spaceland.  If our friend
comes closer to us we see his line becomes larger; if he leaves us it
becomes smaller; but still he looks like a straight line; be he a
Triangle, Square, Pentagon, Hexagon, Circle, what you will--a straight
Line he looks and nothing else.

You may perhaps ask how under these disadvantagous circumstances we are
able to distinguish our friends from one another: but the answer to
this very natural question will be more fitly and easily given when I
come to describe the inhabitants of Flatland.  For the present let me
defer this subject, and say a word or two about the climate and houses
in our country.

\include{1-introduction}
\include{2-related}
\include{3-model}
\include{4-predictions}

\chapter{Conclusions}

We conclude that graduate students like coffee.

\appendix*

\include{Appendix.tex}

\backmatter

\bibliography{thesisbib}

\chapter{Vita}

Juan Valdez was born\ldots.

\end{document}
\endinput
%%
%% End of file `thesis-ex.tex'.
