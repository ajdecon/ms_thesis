\begin{abstract}
The self-assembly of colloidal particles into larger structures is of 
interest both scientifically and technologically. 
The range of possible structures that may be formed by 
isotropically-interacting spherical particles is narrow, encompassing only a few 
possibilities.  To overcome this limitation, one can introduce one or more forms of 
anisotropy to the particles to guide their self-assembly.

In this work, we study the fabrication and behavior of polymeric microparticles 
that are chemically- and shape-anisotropic.
Single-component, rod-shaped particles are fabricated 
by stop-flow lithography (SFL) using either hydrophobic and 
hydrophilic materials. SFL is also used to fabricate
Janus particles that incorporate both 
chemistries within a single particle.  The dynamical
behavior and self-assembly of these rods are investigated using 
fluorescence and confocal microscopy over a range
of different aspect ratios and environmental conditions.  We also developed
image processing algorithms to enable the quantitative analysis of 
these data, adapting standard particle identification and tracking
techniques to the analysis of rod-shaped colloids.
Finally, we demonstrated the fabrication of colloidal particles with 
branched and more complex morphologies, and briefly studied
the self-assembly of these ``patchy'' particles.
\end{abstract}
