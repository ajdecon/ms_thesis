\begin{abstract}
The self-assembly of colloidal particles into larger structures is of 
interest in a wide variety of scientific and practical applications such as
the study of self-assembly~\cite{glotzer-solomon} and the fabrication of 
photonic crystals~\cite{vos-photonic, yang-photonic} and 
three-dimensional templates for tissue-engineering 
scaffolds.~\cite{zhang-tissue}  However,
the range of possible structures that may be formed by the self-assembly
of isotropically-interacting spherical particles is narrow, encompassing only a few different 
possibilities.  To address this limitation and enable the production of 
a wider range of structures, one possibility is to alter the
self-assembly characteristics by introducing particles which incorporate 
one or more forms of 
anisotropy.

In this work, we study the fabrication and behavior of polymeric microparticles 
that incorporate several different forms of anisotropy.
Single-component rod-shaped particles are fabricated 
by stop-flow lithography~\cite{dendukuri-sfl} using either hydrophobic and 
hydrophilic materials. SFL is also used to fabricate
Janus particles that incorporate both 
materials in a single particle.  The dynamical
behavior and self-assembly of these rods are investigated using 
fluorescence and confocal microscopy over a range
of different aspect ratios and environmental conditions.  We also develop
image processing algorithms to enable the quantitative analysis of 
these data, adapting standard particle identification and tracking
techniques to the analysis of rod-shaped colloids.
Finally, we demonstrate the fabrication of Janus particles with 
branched and more complex morphologies, and briefly investigate
the self-assembly of these ``patchy'' particles.
\end{abstract}
