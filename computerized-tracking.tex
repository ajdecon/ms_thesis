\chapter{Computerized Tracking of Anisotropic Colloids}
\section{Introduction}

Confocal laser scanning microscopy (CLSM) is a powerful technique for the study of three-dimensional
structure in fluorescent materials. When applied to fluorescent colloids, CLSM enables the observation
and identification of individual particles, determining their positions in three-dimensional space.
These particle locations alone can be used to derive a great deal
of information about the material, such as the distribution of number of nearest neighbors and
the pair distribution function (PDF). Repeated observations at regular intervals allow for dynamical
measurements of parameters such as the diffusion constant, and may be used to study the microstructual
differences between different parts of the colloidal phase diagram. CLSM has the additional advantage
that since it produces real-space position data, it has substantial advantages over scattering techniques
in coping with samples with highly asymmetric structures.~\ref{?}

However, 
the production of 3D structural information requires more than just a powerful imaging technique: it also
requires powerful computational analysis to translate image data into a list of particles and positions, and
to determine the relevant physical data from this list.  In addition, the behavior of non-spherical colloids
is governed not only by the relative positions of the particles but also their orientations. Developing an
understanding of anisotropic colloids
therefore calls for the development of image processing techniques for the extraction and analysis of
structural data from microscopy images.

\section{Literature review}

\subsection{Particle tracking with spherical colloids}

The current state of the art in 

\subsection{Solomon rod tracking}

\tempfigure{Illustration of Solomon technique}
\tempfigure{SFL particles as 2D extruded objects; flat fluorescence}
\begin{itemize}
\statement{Desirable to locate anisotropic particles in microscopy images to study dynamics and assembly.}
\statement{Previous work in this area: Crocker and Grier, Solomon}
\statement{Can't use Solomon method: fluorescence is too flat for SFL particles. 2D-extruded objects. (Figure to illustrate.)}
\statement{Must develop new algorithms which can work on these particles.}
\end{itemize}

\section{Rod-tracking algorithm}

Our algorithm for locating and tracking SFL rods draws heavily from the algorithm published by 
Mohraz and Solomon for tracking PMMA rods.~\ref{moraz-solomon-rods}.  Briefly, this algorithm 
took advantage of the gradient in fluorescent intensity throughout the volume of the 
rods they studied, which were fabricated by stretching spherical PMMA particles along a single axis.
Because these rods had a circular cross-section, scanning the confocal laser through a point nearer
to the rod axis would pass through a greater volume of fluorescent material, producing a higher intensity.
By applying a local line maximum criterion to the points inside the particle volume, they were able to 
build a ``backbone'' of points near the axis, allowing them to reliably calculate orientation.

While this algorithm performs very well for a restricted class of rods, it fails in cases where the particle
cross-section is not circular, and points near the particle backbone are not guaranteed to produce higher 
intensities than their immediate neighbors.  This is the case for our ``rods'' produced by stop-flow
lithography (SFL), in which the sides of the rods are relatively flat due to the fabrication
geometry. These particles have correspondingly flat fluorescence profiles, and require a more complex analysis
to calculate a ``backbone''.

We have developed an algorithm for processing 2D and 3D CLSM data of fluorescent SFL rods to
produce position and orientaion data.  Starting from raw CLSM images, this algorithm can be divided
into several phases, including (i) image cleanup; (ii) segmentation; (iii) position calculation; and
(iv) time-series calculation.  

\subsection{Image cleanup}

Two different image cleanup methods were used

\tempfigure{Image processing flowchart}
\tempfigure{Example image: cleanup, segmentation, skeleton, track}
\tempfigure{Schematic comparison of 2D/3D}
\tempfigure{Schematic explanation of ``template'' technique for tracking arbitrary shapes}
\tempfigure{Example images for ``template'' tracking}
\begin{itemize}
\done{Tracking algorithm for rods}
\begin{itemize}
\done{Image cleanup using erosion, opening}
\done{Particle segmentation using watershed transform}
\done{Particle skeleton (backbone) using distance transform plus rank-order filter}
\done{Calculation of center-of-mass, oritentation according to Solomon}
\done{Time-series tracking according to Blair and Dufrense}

\done{Difference between 2D and 3D version of algorithm}

\end{itemize}

\notdone{Tracking algorithm for arbitrary shapes}

\begin{itemize}
\done{Initial steps same as for rods: cleanup, segmentation}
\done{``Skeletonization'' as above, but producing a more complex skeleton than for rods}
\done{Calculate center-of-mass as with rods}
\done{Choose a sample skeleton as the ``canonical'' particle skeleton}
\done{Isolate particles into individual windows}
\done{To measure orientation: rotate canonical skeleton image in small increments. For each one, AND together the 
rotated skeleton and the sample skeleton.  Maximize pixel sum of ANDed image.}
\done{Faster in 2D than in 3D}
\notdone{Implementation not finished!}
\end{itemize}

\end{itemize}

\section{Implementation}

\tempfigure{Example particle tracks}
\begin{itemize}
\done{Rod tracking implemented in 2D in Matlab.}
\statement{Go over the details of the matlab implementation}
\notdone{Rod tracking implemented in 3D in Matlab, but with major bugs.}
\notdone{Arbitrary tracker not yet fully implemented, using Matlab.}
\end{itemize}

\section{Assessment}

\tempfigure{Cartoons to show possible errors}
\begin{itemize}
\done{Identify issues caused by morphological image processing for these algorithms.}
\notdone{Estimate errors in particle segmentation}
\notdone{Estimate error of rod COM and orientation calculations.}
\notdone{Estimate error of skeleton-based orientation calculations.}
\end{itemize}
