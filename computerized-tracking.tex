\chapter{Computerized Tracking of Anisotropic Colloids}
\section{Introduction}

\tempfigure{Illustration of Solomon technique}
\tempfigure{SFL particles as 2D extruded objects; flat fluorescence}
\begin{itemize}
\statement{Desirable to locate anisotropic particles in microscopy images to study dynamics and assembly.}
\statement{Previous work in this area: Crocker and Grier, Solomon}
\statement{Can't use Solomon method: fluorescence is too flat for SFL particles. 2D-extruded objects. (Figure to illustrate.)}
\statement{Must develop new algorithms which can work on these particles.}
\end{itemize}

\section{Algorithms}

\tempfigure{Image processing flowchart}
\tempfigure{Example image: cleanup, segmentation, skeleton, track}
\tempfigure{Schematic comparison of 2D/3D}
\tempfigure{Schematic explanation of ``template'' technique for tracking arbitrary shapes}
\tempfigure{Example images for ``template'' tracking}
\begin{itemize}
\done{Tracking algorithm for rods}
\begin{itemize}
\done{Image cleanup using erosion, opening}
\done{Particle segmentation using watershed transform}
\done{Particle skeleton (backbone) using distance transform plus rank-order filter}
\done{Calculation of center-of-mass, oritentation according to Solomon}
\done{Time-series tracking according to Blair and Dufrense}

\done{Difference between 2D and 3D version of algorithm}

\end{itemize}

\notdone{Tracking algorithm for arbitrary shapes}

\begin{itemize}
\done{Initial steps same as for rods: cleanup, segmentation}
\done{``Skeletonization'' as above, but producing a more complex skeleton than for rods}
\done{Calculate center-of-mass as with rods}
\done{Choose a sample skeleton as the ``canonical'' particle skeleton}
\done{Isolate particles into individual windows}
\done{To measure orientation: rotate canonical skeleton image in small increments. For each one, AND together the 
rotated skeleton and the sample skeleton.  Maximize pixel sum of ANDed image.}
\done{Faster in 2D than in 3D}
\notdone{Implementation not finished!}
\end{itemize}

\end{itemize}

\section{Implementation}

\tempfigure{Example particle trackes}
\begin{itemize}
\done{Rod tracking implemented in 2D in Matlab.}
\statement{Go over the details of the matlab implementation}
\notdone{Rod tracking implemented in 3D in Matlab, but with major bugs.}
\notdone{Arbitrary tracker not yet fully implemented, using Matlab.}
\end{itemize}

\section{Assessment}

\tempfigure{Cartoons to show possible errors}
\begin{itemize}
\done{Identify issues caused by morphological image processing for these algorithms.}
\notdone{Estimate errors in particle segmentation}
\notdone{Estimate error of rod COM and orientation calculations.}
\notdone{Estimate error of skeleton-based orientation calculations.}
\end{itemize}