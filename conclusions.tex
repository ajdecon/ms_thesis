\chapter{Conclusions}
\label{ch:conclusions}

In this thesis, we studied the fabrication, self-assembly, and dynamical behavior
of shape- and chemically-anisotropic colloidal particles.  The fabrication 
of hydrophilic, hydrophobic, and Janus colloidal rods of various sizes and aspect ratios was demonstrated via
stop-flow lithography.  Image processing algorithms
were developed for identifying, locating, and measuring the orientation of 
colloidal rods in microscopy images, and software was implemented to carry out
this analysis and characterize structural and dynamical properties.
Single-component particle 
dynamics were observed via time-series confocal microscopy using particles of 
various aspect ratios to study the effects of particle geometry on diffusion.
The self-assembly of Janus rods was studied using
3D confocal microscopy for particles of various aspect ratios
in a number of different solvent conditions to determine the effects of size 
and attraction strength on the resulting structure.  The fabrication of more complex
multiple-patch particles with several different geometries was also demonstrated via SFL.

\section{Future Work}

Significant challenges were identified in the areas of particle collection and processing as 
detailed in Section~\ref{sec:rod-collection}. These complications may have significantly affected the 
results of dynamics and self-assembly experiments.  It may be possible to reduce these
processing complications with the identification of 
an appropriate non-stick coating.  Another possible improvement would be the use of an
integrated microfluidic concentration
and cleaning solution to exchange the monomer solution for another solvent.

Once these issues have been solved, a number of interesting studies on the structure and
dynamics of suspensions of microfabricated colloids are possible.  These include studying 
self-assembly and dynamics of colloidal suspensions at high concentrations as well as 
the behavior of two-component suspensions containing a mix of particles with different geometries and
functionalities.

It should also be possible to develop image processing algorithms to track particles with more
complex geometries.  The algorithm developed in Chapter~\ref{ch:comp-tracking} is independent 
of particle geometry throughout the image cleanup, segmentation, and skeletonization steps.
The development of algorithms which could translate the skeletons of other particle 
geometries into position and orientation information would greatly expand the 
capabilities of particle tracking.
