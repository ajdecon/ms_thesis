\chapter{Conclusions}
\label{ch:conclusions}

In this work we studied the fabrication, self-assembly and dynamical behavior
of shape- and chemically-anisotropic colloidal particles.  The fabrication 
of colloidal rods of various sizes and aspect ratios was demonstrated via
stop-flow lithography for rods with
hydrophilic, hydrophobic and Janus functionalities.  Image processing algorithms
were developed for identifying, locating and measuring the orientation of 
colloidal rods in microscopy images, and software was implemented to carry out
this analysis and characterize structural and dynamical properties.
Single-component particle 
dynamics were observed via time-series confocal microscopy using particles of 
various aspect ratios to study the effects of particle geometry on diffusion.
The self-assembly of Janus rods was studied using
3D confocal microscopy for particles of various aspect ratios,
in a number of different solvent conditions, to determine the effects of size 
and attraction strength on the resulting structure.  The fabrication of more complex
multiple-patch particles with several different geometries was also demonstrated via SFL.

\section{Future Work}

Significant challenges were identified in the areas of particle collection and processing as 
detailed in Section~\ref{sec:rod-collection} which may have significantly affected the 
results of dynamics and self-assembly experiments.  It might be possible to solve these
processing issues using some non-stick coating instead of the fluorosilane coating 
which was used.  Another avenue to explore includes the use of a microfluidic concentration
and cleaning solution to exchange the monomer solution for another solvent, 
which may be integrated with the fabrication system.  

Once these issues have been solved, a number of interesting studies on the dynamics and
structure of suspensions of microfabricated colloids are possible.  This includes studying 
self-assembly and dynamics of colloidal suspensions at high concentrations, as well as 
the behavior of biphasic suspensions containing a mix of particles with different geometries and
functionalities.

It should also be possible to develop image processing algorithms to track particles with more
complex geometries.  The algorithm developed in Chapter~\ref{ch:comp-tracking} is independent 
of particle geometry throughout the image cleanup, segmentaton and skeletonization steps.
The development of algorithms which could translate the skeletons of other particle 
geometries into position and orientation information would greatly expand the 
capabilities of particle tracking.
