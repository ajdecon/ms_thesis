\chapter{Rod tracking in Matlab}
\label{sec:matlab-implementation}

The results of typical experiments with samples of single-component or Janus rods included both 
large-area tiled images taken of many particles at a single time,
used to study static self-assembly, and movies consisting of many subsequent images in a 
particular location, taken to study dynamical behavior.  Each of these experimental types required the processing
of image sets numbering in the hundreds or thousands.  The analysis of these images using the algorithm developed
above requires the selection of a number of input parameters, such as the thresholding levels and the 
structuring elements for morphological processing.  However, while all the images from a typical
experiment could be expected to share the same parameter values, those values might vary considerably for the 
analysis of different
experiments.

To address this, we implemented our algorithm as a set of independent functions which could be carried out manually 
or called from an automated script. A typical analysis was carried out by selecting one or more 
test images from the data set; carrying out the various image processing steps on these test images, varying
the processing parameters to obtain the best results; and then calling the automation script using the 
optimized values to process the entire data set.  Analysis of test images was generally carried out on a single
workstation, while full-dataset processing was carried out on dedicated servers to maximize processing efficiency.

All steps of the analysis were implementing in Matlab~\cite{matlab}, making heavy use of functions from the
Image Processing Toolbox.  The following description summarizes the process of carrying out a manual analysis of
test images as a guide to future users.

\begin{lstlisting}[label=ls:manual,caption=Typical sequence of a manual analysis]
% Load the image.
img = imread('filename.tif');

% If using bandpass cleaning.
clean = bpass(img,lnoise,lobject);

% If using morphological cleaning.
%  top-hat step: radius 50 is greater than rod size.
%  opening step: radius 2 is a good size for eliminating small variations.
thstruct = strel('disk',50);
opstruct = strel('disk',2);

% Segmentation step.
watershed_labels = segment(clean, 1);

% Now find backbone pixels.
skeletons = backbones(clean, watershed_labels, 90);

% Finally get the list of positions.
positions = calc_positions(skeletons);
\end{lstlisting}

\section{Image Cleanup}

Image cleanup is probably the part of the analysis which is most sensitive to parameter selection.
Original images of colloidal rod samples from CLSM or FM are often noisy or unevenly illuminated, and these
effects vary from experiment to experiment.  However, subsequent steps of the analysis assume that their input
images will be simple binary images, with white rods and a black background. Choosing the correct parameters 
to produce such images is a matter of trial and error, and the particular choices must be re-optimized
for each new experiment.  Two options exist for performing this clean-up: a simple bandpass filter, 
and a more complex set of morphological operations.

\subsection{Option 1: Band-Pass}

\texttt{clean = bpass(image\_array, lnoise, lobject[, threshold])}

\begin{itemize}
\item \texttt{image\_array}: Matlab array containing image pixels.
\item \texttt{lnoise}: Characteristic length-scale of noise.
\item \texttt{lobject}: Characteristic length-scale of object to be tracked (i.e., length of a colloidal rod).
\item \texttt{threshold}: By default, the last step of this algorithm is to set all negative pixels (generated
by the convolution) to be zeros. This parameter may optionally be used to apply a threshold with a different cut-off.
\end{itemize}

\texttt{bpass.m} is from the Blair and Dufrense~\cite{blair-dufrense-matlab} Matlab implementation of the Crocker and Weeks
package for tracking of spherical particles.~\cite{crocker-grier-spheres}

\subsection{Option 2: Morphological Cleanup}

\texttt{clean = mclean(img, thstruct, opstruct[, threshold])}

\begin{itemize}
\item \texttt{img}: Matlab array containing image pixels.
\item \texttt{thstruct}: Matlab structuring element used to carry out the top-hat transform.
\item \texttt{opstruct}: Matlab structuring element used to carry out the opening and dilation operations.
\item \texttt{threshold}: By default, the threshold used in this algorithm is selected using the built-in
Matlab function \texttt{graythresh}, which uses Otsu's algorithm~\cite{otsu-threshold}. Here, the user may optionally
select a different threshold.
\end{itemize}


The morphological method for image cleanup is implemented in \texttt{mclean.m}, and consists of a top-hat transform,
a thresholding, a binary opening, a dilation, and a bitwise AND operation performed in sequence.  
The function expects the structuring elements for the morphological operations
to be passed as arguments, produced by the 
\texttt{strel} function.  This is a 
highly computationally-intensive sequence of operations, but is easily produced using the the functions provided
by the Matlab Image Processing Toolkit.  The sequence of operations, in Matlab, is illustrated in 
Listing~\ref{ls:mclean}.

\begin{lstlisting}[label=ls:mclean,caption=Morphological image cleanup]
% Even out illumination using top-hat transform.
th = imtophat(img,thstruct);

% Threshold to produce a binary image.
bw = im2bw(th, graythresh(th));

% Morphological operations: open and dilate to remove noise.
mo = imdilate(imopen(bw,opstruct),opstruct);

% Produce final image with AND mask.
clean = img;
clean(mo==0) = 0;
\end{lstlisting}

\section{Segmentation}

\texttt{watershed\_img = segment(img[, height])}

\begin{itemize}
\item \texttt{img}: Matlab array containing the image pixels.
\item \texttt{height}: Maximum height to suppress in h-minima transform. Optional, defaults to 1.
\end{itemize}


\texttt{segment.m} carries out the image segmentation part of the algorithm, and consists of three calls to
Matlab built-in functions: \texttt{bwdist}, which carries out the distance transform, 
\texttt{imhmin}, which carries out the h-minima transform, and \texttt{watershed}, which performs watershed
segmentation. While these processes are all computationally intensive, the result is relatively insensitive to
processing parameters. The only parameter available is the height of the h-minima transform, which is generally
set to 1 to account for single-pixel fluctuations; 
it is increased only when over-segmentation is observed in the resulting watershed.~\cite{matlab-watershed}
To observe the resulting watershed segmentation, use the code in Listing~\ref{ls:showwater}.

\begin{lstlisting}[label=ls:showwater,caption=Observe watershed segmentation]
imshow( label2rgb(watershed_img,'jet') );
\end{lstlisting}

\section{Skeletonization}

\texttt{skeletons = backbones(img, watershed\_img, percent)}

\begin{itemize}
\item \texttt{img}: Matlab array containing the image pixels.
\item \texttt{watershed\_img}: Image containing the watershed labels.
\item \texttt{percent}: Percentile used in the thresholding step.
\end{itemize}

The generation of the rod skeletons, carried out in \texttt{backbones.m}, is also a relatively simple
procedure.  The results of the watershed segmentation step are used to find the portion of the 
image which contains each rod, and this is used to generate a new image in which the rod may
be analyzed in isolation. The only choice which must be made is the percentile threshold for selecting 
backbone pixels. This is again a matter of trial and error, but typical values are in the range of 90-95\%.

A final image is generated which contains all the backbones, with the pixels belonging to each one having
the value of their watershed label. This allows them to be uniquely identified in the following step. The background is 
assigned again to zero.  Observation of these backbones for quality check requires a thresholding step. Observation
of all the backbones may be accomplished using the code in Listing~\ref{ls:allbb}, while observing only the backbone 
with label $n$ may be accomplished using the code in Listing~\ref{ls:onebb}.

\begin{lstlisting}[label=ls:allbb,caption=Show all backbones as an image]
imshow(im2bw(skeletons, 0));
\end{lstlisting}

\begin{lstlisting}[label=ls:onebb,caption=Show only backbone with label $n$]
temp=skeletons;
temp(temp~=n)=0;
imshow(im2bw(skeletons, 0));
\end{lstlisting}

\section{Coordinate Calculation}

\texttt{positions = calc\_positions(skeletons[, cutoff])}

\begin{itemize}
\item \texttt{skeletons}: Image containing the rod skeletons.
\item \texttt{cutoff}: Optional parameter listing a cut-off for ignoring a backbone.
\end{itemize}

For each individual rod, \texttt{calc\_positions.m} calculates the positional and
orientational coordinates and saves them to the array \texttt{positions}.  
\texttt{cutoff} is an optional parameter which allows \texttt{calc\_positions.m} to 
ignore backbones which contain under a certain number of pixels, as one last noise-protection step.

\texttt{positions} is a 2D Matlab array in which each row represents one backbone, and has 
the structure:

\begin{table}[h]
\begin{center}
\begin{tabular}{ | c | c | c | c | c | }
\hline 
$x$ & $y$ & $z$ & $\phi$ & $\theta$ \\
\hline
\end{tabular}
\end{center}
\caption{A single row of the \texttt{positions} data structure, identifying the coordinates of one rod.}
\label{tab:positions}
\end{table}

In any 2D image, $z$ and $\theta$ are always zero.

\section{Automated Analysis of Time Series}

\texttt{poslist = rod\_tseries\_bp(imgstack,lnoise,lobject,threshold,height,percent)}

\texttt{poslist = rod\_tseries(imgstack,thstruct,opstruct,threshold,height,percent)}

\section{Temporal Tracking}

\texttt{tracks = track(xyzs, maxdisp, param)}

\begin{itemize}
\item \texttt{xyzs}: An array containing a time-sorted list of particle positions (and, here, orientations).
\item \texttt{maxdisp}: Maximum allowed displacement of a particle between frames.
\item \texttt{param}: A data structure containing additional processing parameters.
\end{itemize}

Frame-to-frame tracking of unique rods to form trajectories was accomplished using a Matlab routine, \texttt{track.m}, 
supplied by Blair and Dufrense~\cite{blair-dufrense-matlab}, implementing the standard algorithm for particle tracking by
Crocker and Weeks~\cite{crocker-grier-spheres}.  \texttt{track.m} requires that the data be in a time-sorted format where each
row consists of a list of coordinates followed by a frame number, and the frame numbers increase monotonically.

The output, \texttt{tracks}, is in a similar format which includes one extra column: a particle ``id number'' which
allows each trajectory to be identified. Rows are ordered so that particle id number increases monotonically, and
time increases monotonically within each set of particle rows.

The optional input structure, \texttt{param}, contains a variety of settings which alter the behavior of 
\texttt{track}.  The only setting important to this analysis is \texttt{param.dim}, which tells the program how
many columns to use as positional dimensions. Any additional columns will be ignored in the tracking calculation and 
simply ``carried along'' when the new array is built; this gives us a place to put our orientation coordinates, which
will not be used in the tracking routine.

\section{Characterization of Rod Suspensions}

The location and tracking of colloidal rods within experimental images is not an end in itself, but the
first step in determining the characteristics of the suspensions they are used to measure.  Data on the 
position and orientation of all rods within a suspension is an extremely useful tool, and the values of many
dynamical or structural properties may be directly calculated or inferred from this information.
While the current work did not proceed so far as to complete a detailed study of all these properties, or 
implementations of the calculations necessary for such a study, some preliminary work has been done in this
area which is worth exploring.

\subsection{Dynamics}

%\texttt{traj\_vis(tracks[, t0, t1])}

\texttt{showmovie(imgstack,tracks)}

%\texttt{traj\_vis.m} is a simple diagnostic tool which plots the trajectories for all the particles. 
%\texttt{t0} and \texttt{t1} are optional beginning and ending times.  
\texttt{showmovie.m} visualizes the
movie from the image stack, plotting a dot-and-bar for each tracked particle on the image to indicate 
the tracked position and orientation.

\texttt{disps = msd(tracks)}

\texttt{avgs = avg\_msd(disps)}

\texttt{msd.m} calculates the mean-square displacement (MSD) for position and orientation for each particle 
in \texttt{tracks}, and returns a data structure \texttt{disps} which contains each individual MSD vs time 
series.  \texttt{avg\_msd.m} takes this structure and averages across particles to produce plots similar 
to Figure~\ref{fig:example-diffusion-results}. Note that this averaging takes place at each 
individual particle's \textit{own} frame 2, 3, etc. so that the displacement at frame 2 of two particles 
will be averaged together--even if the second particle did not appear until a later frame.  This reduces the
total time over which averaging may take place, but improves statistics.

\subsection{Structure}
\label{sec:structure-calcs}

\texttt{result = nearest\_neighbors(positions, size)}

%\texttt{pdf = pair\_dist(positions)}

\texttt{result = orient\_corr(positions)}

\texttt{nearest\_neighbors.m} calculates the average number of nearest neighbors for the particles in 
\texttt{positions} within a distance given by \texttt{size}. 
%\texttt{pair\_dist.m} produces the pair-distribution
%function (PDF) for all particles in \texttt{positions}, where as 
\texttt{orient\_corr.m} shows how the 
dot-product between particle orientations changes with distance.

