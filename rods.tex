\chapter{Assembly and Dynamics of Rod-Shaped Colloids}
\section{Introduction}

When one is first presented with the different dimensions of anisotropy proposed by 
Glotzer and Solomon (Figure~\ref{fig:glotzer-dimensions},\ref{glotzer-solomon-assembly}), the
potential variety of particle types can be overwhelming. It is therefore useful to begin by studying 
particles which draw from only one or two of these anisotropy dimensions. To this end, our
initial study has focused on particles which vary only in the dimension of 
aspect ratio, i.e. colloidal rods. This study covers the fabrication of both single-component and 
Janus rods by stop-flow lithography, the study of rod diffusion by particle tracking, and the basics
of self-assembly for Janus rods in different solvents.

%\tempfigure{Colloidal rod examples}
%\begin{itemize}
%\statement{Natural rod systems}
%\statement{Systems studied by Solomon}
%\statement{Janus colloids: Granick}
%\statement{Interest in Janus rods}
%\end{itemize}

\section{Experimental Procedure}

Colloidal rods were fabricated by stop-flow lithography (SFL) as described in section~\ref{sec:SFL} using
hydrophobic and hydrophilic monomer solutions.  

\subsection{Microchannel device fabrication}

\figone{fig:device-design}{figures/rods/four-channels-together.jpg}{\linewidth}{
Four Y-channel microchannel patterns, sharing a common reservoir for accumulation of particles from multiple experiments.}

Particle fabrication was carried out in Y-junction microchannel devices (Figure~\ref{fig:device-design}).
The primary channel of these devices had a typical 
height of 7 \microns, width of 200 \microns, and length of 3000-5000 \microns. These dimensions were selected
to facilitate the fabrication of large numbers of particles with small size in all dimensions: the low height
facilitated small-particle fabrication by limiting the height of the particles, while the comparatively large
width allowed many particles to be fabricated simultaneously. Multiple entrances were defined to allow up to three
monomer streams to be simultaneously flowed, with a single exit point for collecting particles.

\subsubsection{Photoresist masters}
Positive-relief photoresist master templates were fabricated by UV photolithography. A thin film
of SU-8 2007 photoresist (Microchem) was laid down on a clean Si wafer via spin-coating at 3000 rpm to produce a 
7 \microns layer. Next, a ``soft bake'' was carried out by heating the wafer on a hot plate at 120\degC
for five minutes to evaporate the photoresist solvent.  The device features were patterned by exposing the 
photoresist to UV light ??? for 40 s through a photomask defining the device design. A ``hard bake'' step
was then carried out by heating the wafer at 120\degC for ten minutes, to cure the photoresist in the exposed areas.
Finally, the wafer was immersed in SU-8 developer (Microchem) and agitated for two minutes to remove the unexposed
photoresist, then rinsed with isopropyl alcohol (IPA).

After fabrication, photoresist masters were subjected to a fluorinated silane vapor coating to inhibit adhesion
between the SU-8 template and the elastomer to be cast. Masters were placed in a small desiccator (Fisher Scientific)
along with an open container of (tridecafluoro-1,1,2,2-tetrahydrooctyl) trichlorosilane (Gelest, Inc.)
This desiccator's vacuum port was then connected to a single-stage vacuum pump and evacuated for two hours to produce
a silane coating.

\subsubsection{Elastomer device construction}

Microchannel devices were constructed from polydimethylsiloxane (PDMS, Dow Corning, Sylgard 184). PDMS
elastomer and curing agent were mixed at a ratio of 10:1 by weight, and pored over the photoresist master 
in a plastic petri dish to a depth of about 2 mm.  PDMS was also spun-coat onto a 48 x 60 mm \#1 cover-slip (Gold Seal)
to form the substrate for the device.
Both of these  were then baked at 65\degC for six hours or more to cure
the PDMS.  

Once the PDMS was fully cured, a razor blade was used to carefully cut out a section which encompassed some or all of
the microchannels defined on the photoresist master. This section was then peeled up from the master, revealing
a block of PDMS which contained negative features defining the top and sides of the microchannels.
For each microchannel, three small holes (\~ 0.5 mm) were punched at each entrance using a syringe press, and a larger
hole (\~ 3 mm) was punched at the exit using a biopsy punch.

For each of the ``top side'' PDMS block and the PDMS-coated glass substrate, the PDMS surface was rinsed with 
deionized water and IPA.  Following this, small particles were removed by first laying down and then peeling up
Scotch Magic brand transparent tape.~\ref{rogers-tape-ref}  Each section was then placed below a UV light-emitting
diode with the channel surface facing the diode, and exposed to UV light for ten minutes to promote PDMS-PDMS
adhesion.  After UV exposure, these sections are then firmly pressed together with the channel surface of the top
block against the PDMS-coated surface of the substrate.  The resulting device is then baked at 100\degC for one hour
to promote device bonding.

\subsection{Materials}
The hydrophobic solution was composed of 95 v/o 
tri(methylol propane) triacrylate (TMPTA, Sartomer) and 5 v/o Darocur 1173 photoinitiator (Ciba), 
with 0.005 wt\% methacryloxyethyl thiocarbamoyl rhodamine B (Polysciences) as a cross-linking 
fluorescent dye.
The hydrophilic solution was composed of either 20 mol ethoxylated tri(methylol propane) triacrylate (20-ETMPTA,
Sartomer) or poly(ethylene glycol) diacrylate (PEGDA, $M_n$ = 700, Sigma Aldrich) at 80 v/o, 
15 v/o deionized water, and 5 v/o Darocur 1173 photoinitiator, with 0.005 wt\% 
3,8-dimethacryloyl ethidium bromide (Polysciences) as a cross-linking fluorescent dye.


\subsection{Mask design}

Masks used for single-component fabrication contained two-dimensional arrays of identical aligned 
rods, with a separation in each direction equal to twice the length of the rod to avoid inter-particle
curing (Figure~\ref{fig:rod-masks}(a)). These arrays were designed to maximize the number of 
rods cured per cycle by making them large enough to, at minimum,
cover the field stop aperture for the transmission of the UV beam. This circular aperture 
had a diameter of 1.5 inches.  For example, the photomask containing 500~\microns rod features
was a twenty-by-twenty array, with the 1 mm separation ensuring that the mask area was large enough to
use the full available beam.  
Masks used for Janus fabrication contained only a single line of rod features, with the rods parallel to one 
another and aligned perpendicular to the axis of the line (Figure~\ref{fig:rod-masks}(b)).  Spacing on these 
masks was the same as for single-component fabrication.


\subsection{SFL experiment}

\figone{fig:sfl-experiment-photo}{figures/rods/microchannel-experiment.JPG}{\linewidth}{
A microfluidic device platform containing multiple Y-junction microchannels is placed on 
the microscope used for SFL, and connected to two pressure sources to pump PEGDA and TMPTA 
monomer solutions.}

UV exposure and experimental imaging for small-rod fabrication was carried out using a 60x 
oil-immersion objective lens (Olympus America), with an additional 1.6x lens added to the beam path for
the fabrication of smaller rods.  Using the 60x objective, a demagnification factor of approximately 
33 was typically observed between the mask and the resulting rods; i.e., a rod of 500~\microns length defined on
the photomask would typically result in the fabrication of a 12~\microns rod in the microchannel.

Microchannel flow was driven by gas pressure supplied by a house nitrogen line or a compressed air tank 
(SJ Smith Welding Supply). Pressure control was achieved using a custom-built pressure box, consisting of
four computer-controlled regulators and four duplex valves connected to a USB
controller (National Instruments) allowing for 
up to four independent pressures driving up to eight separate lines. UV light was supplied by a ?? W mercury lamp
connected in fluorescence microscope configuration, with exposure time controlled using a Lambda SC 
electronic shutter (Sutter Instruments).

The SFL experiment was driven using LabView software (National Instruments) with a custom
interface (Figure~\ref{fig:sfl-labview}). The process
flow of an SFL experiment is illustrated in Figure~\ref{fig:sfl-flowchart}, and consists of the following steps:

\begin{enumerate}
\item Specify ``flow time'' $t_{flow}$, ``pause time'' $t_{pause}$, and exposure time $t_{expose}$.
\item Set pressure for each port on electronic regulators.
\item Until the experiment is stopped:
\begin{enumerate}
\item Open valves. Wait $t_{flow}$.
\item Close valves. Wait $t_{pause}$.
\item Open shutter. Wait $t_{expose}$.
\item Close shutter. Repeat.
\end{enumerate}
\end{enumerate}

For a schematic of the SFL experimental setup, see Figure~\ref{fig:sfl-schematic}.
For each monomer solution, a 10-100\uL pipette tip (Eppendorf) was filled with approximately 60\uL of solution.
A silicone tube was connected at one end to the output of one valve of the pressure box, and the other end
was inserted into the top of the pipette tip.  The bottom of the pipette tip was then inserted into one of the
small entrance holes in the microchannel device.  This device was the placed on above the microscope objective,
and the microscope was focused such that the focal place was inside the channel.  After starting the 
experiment, the shutter was allowed to open and the fine focus was adjusted manually to optimize
curing conditions. Optimization was judged by visual inspection of the resulting particles. 

Single-component rods were fabricated using the central entrance channel to supply monomer to the system, 
leaving the other two entrances unused. Typical experimental settings for these experiments were pressure
8 psi, flow time 1.5 s, and pause time 2 s.  Exposure times for PEGDA or 20-ETMPTA were typically 0.05-0.15 s
depending on the mask dimensions, while exposure times for TMPTA were typically 0.2-0.3 s to account for
the less favorable curing behavior of the hydrophobic monomer.  The fabrication of smaller rods generally
necessitated slightly higher exposure times.  

Janus rods were fabricated using the two side entrance channels to supply each type of monomer solution,
with the central channel left unused.  (All three channels were used for fabricating branched Janus particles; see
section~\ref{sec:SFLx3}.)  Typical experimental settings for these experiments were pressure 7 psi on each side, 
flow time 2 s, pause time 2 s, and exposure time 0.2-0.3 s depending on the mask dimensions.
The two monomer streams would enter the main channel on either side, and form
an interface along the center of the channel (Figure~\ref{fig:Janus-channel}). Because the hydrophobic and
hydrophilic streams were immiscible and the microchannel flow took place in the laminar flow regime~\ref{?}, 
this interface persisted for several hundred microns down the channel.
During fabrication, the curing regions would be aligned such that each rod straddled this interface,
producing rods which incorporated both materials.

In either case, the resulting particles were ejected at the end of the experiment into the final reservoir, which
was left empty to allow collection of the maximum volume of particles possible.


\tempfigure{Photo of SFL setup; SFL flowchart; example experiment}
\begin{itemize}
\done{Microscope setup}
\done{Pressure system--Rob}
\done{LabView controller for SFL}
\done{Design of microfluidic devices for up to 3-stream SFL}
\done{Fabrication of microfluidic devices}
\end{itemize}


\subsection{Particle collection and purification}

Particle collection techniques were developed and optimized for a number of different experimental cases,
depending both on the particular monomer solutions involved and on the desired final rod behavior, i.e. 
free diffusion or the self-assembly of the hydrophobic or hydrophilic material.  Details of the differences
in collection techniques and their effectiveness will be explored in Section~\ref{sec:rod-collection-results}.

In general, SFL-fabricated rods were ejected, still in monomer solution, from the fabrication microchannel into 
an open reservoir, with the bottom and sides formed of PDMS and the top open to the environment.  Owing to the
surface tension of the monomer, these particles generally stayed near the side walls of the reservoir, often
collecting in corners or imperfections of the wall.  Note that in the single-component case, the particles in
the reservoir may be assumed to be perfectly non-interacting as they are still suspended in the monomer solution, 
a solvent with the
same composition as the particles.   In the case of Janus rods this assumption is not necessarily valid, as
the particles are suspended in a mixture of the two different monomer solutions.  Fabrication experiments were
generally stopped well before the monomer solutions could fill the reservoir; an hour-long SFL fabrication experiment
may be expected to use a total volume of monomer solution equivalent to only 1/10th the volume of the
reservoir.  Once fabrication 
was complete, the collection procedure generally consisted of the following steps:

\begin{enumerate}
\item A \textit{collection solvent} was introduced into the reservoir via the top opening, with volume sufficient to
fill the reservoir (generally $\sim$ 20\uL).
\item A slight delay was allowed for the particles to disperse into the new solvent from their 
original position along the 
walls. This delay varied depending on the solvent viscosity, but was generally between 30 and 120 s.
\item A pipette was positioned in the reservoir with the tip near the highest concentration of particles, estimated
by eye using the microscope.
\item The pipette was used to collect a volume equivalent to the amount of collection solvent introduced.
\item Repeat 3-5 times.
\end{enumerate}

The particular solvent used, the dispersal time, and the type of pipette were all varied for different experimental
cases.

Following pipette collection, the particles were either transferred into a micro-centrifuge tube for solvent 
exchange, or directly into an observation chamber.  Solvent exchange was carried out in cases where the desired
observation solvent was either incompatible with PDMS, such as toluene, or of sufficiently high viscosity to make
direct collection in this solvent difficult, such as collecting in PEGDA.  Solvent exchange was also used to reduce
the concentration of the monomer solution components in the final solution, i.e. to remove liquid monomer and 
reduce the background fluorescence due to free dye.

To carry out solvent exchange, particles in a micro-centrifuge tube were either centrifuged at 3000 rpm for 10 minutes or
allowed to settle due to gravity for 6+ hours in order separate particles from solution.  A quantity of solution equivalent
to one-half the total volume was then carefully pipetted from the top of the tube, and replaced with the same quantity of 
the desired \textit{observation solvent}.  The tube was then agitated using a vortex mixer in order to mix the solvents and
re-suspend the particles.  This process was planned to be repeated seven times in order to increase the volume fraction of
the new observation solvent to over 99\%.  However, experimental results indicating that this procedure was resulting in heavy
particle losses resulted in some experiments only involving three or four rinse cycles.  These results will be detailed in 
Section~\ref{sec:rod-collection-techniques}.

\mnote{Add better details on glass observation chambers. Also explain constraints on observation chambers.}

\figone{fig:confocal-chamber-cartoon}{figures/rods/confocal-chamber-cartoon.png}{0.5\linewidth}{
Design of a glass observation chamber for microscopy of particles.}

Observation chambers were constructed by affixing a glass coverslip to one end of short glass tube using 
five-minute epoxy (3M).  Oriented with the coverslip as the base, this provides a flat transparent surface
for microscopy. (See Figure~\ref{fig:confocal-chamber-cartoon}


\subsection{Diffusion Measurements}

To carry out diffusion measurements, the rod sample was placed in the solvent of interest and pipetted into an observation 
chamber.  In some cases the observation chamber was pre-coated with no-stick silane ((tridecafluoro-1,1,2,2-tetrahydrooctyl) 
trichlorosilane, Gelest, Inc.) to inhibit sticking between the rods and the chamber surfaces; in others,
the observation chamber was partially pre-filled with solvent in order to pre-treat the 
bottom surface, and the rods were allowed to settle for 20-30 minutes before observation.

The observation chamber was then placed on the sample stage of a fast confocal laser scanning microscope (CLSM; 
an IX71 inverted microscope, Olympus, Inc. connected to a vt$^{eye}$ confocal scanner, Visitech International). The CLSM
included two independent excitation laser lines with peak wavelengths of 491 nm and 561 nm, corresponding to our ethidium 
bromide and Rhodamine B dyes.  A 100x oil-immersion objective lens was used for imaging the rods which had settled at the bottom
of the chamber.  The microscope was initially used in bright-field imaging mode and manually adjusted to find the particles and 
optimize the focus.

Following the bright-field adjustments, the microscope was put into confocal mode, the laser line selected 
based on the particular sample, and two-dimensional rod diffusion at the 
bottom of the chamber was imaged in time-series mode.  The microscope was configured such that images were acquired at
a rate of 1 frame per second (fps), consisting of the average of eight frames acquired at 30 fps.  
(For fast diffusion of the smallest rods,
image acquisition was adjusted to 2 fps.)  Laser exposure was triggered only during active 
acquisition, and shuttered otherwise.  Image spatial resolution was 512 $\times$ 512 pixels with a scale of 10 pixels
per \microns, and each pixel's intensity was scaled over eight bits to assume values in the range of 0--255.
The intensity scale was adjusted relative to the actual intensity to maximize the dynamic range for the particular sample.
The total time of a typical experiment was 10 minutes.

Time-series image data were saved as 8-bit TIFF image stacks and transferred to group computation servers for analysis.
Analyses were carried out to compute mean-square-displacement and orientation displacement for each rod using algorithms as 
described in Section~\ref{sec:rod-tracking} and implemented in Matlab scripts described in Section~\ref{sec:rods-implementation}, 
and these data
were used to calculate two-dimensional diffusion constants.  



\begin{itemize}
\done{Confocal microscopy setup}
\done{Space and time resolution requirements}
\done{Fluorescence requirements}
\end{itemize}

\section{Results and Discussion}

\tempfigure{Resolution test mask; image of resulting particles}
\subsection{Resolution}
\begin{itemize}
\done{Limiting factors for SFL resolution: mask design, optics, flow effects}
\notdone{Resolution limits constrained by 60X lens--need to redo this (~1 hour work)}
\done{Resolution limits for Janus rods--interface effects}
\end{itemize}

\subsection{Processing techniques}
\label{sec:rod-collection-techniques}


\tempfigure{Image of rod tracking}
\tempfigure{Plots of translational and rotational diffusion data}

\subsection{Translational and Rotational Diffusion}

\figfour{fig:rods-msd}{figures/rods/dynamics-data/msd6um.png}{figures/rods/dynamics-data/msd9um.png}{figures/rods/dynamics-data/msd12um.png}{figures/rods/dynamics-data/msd15um.png}{Mean-square displacement plots for (a) 6 \microns~rods, (b) 9 \microns~rods, (c) 12 \microns~rods and 15 \microns~rods.}

\figfour{fig:rods-rot}{figures/rods/dynamics-data/rot6um.png}{figures/rods/dynamics-data/rot9um.png}{figures/rods/dynamics-data/rot12um.png}{figures/rods/dynamics-data/rot15um.png}{Mean-square rotational displacement data plots for (a) 6\microns~rods, (b) 9 \microns~rods, (c) 12 \microns~rods and 15 \microns~rods.}

\tempfigure{Show fabricated Janus rods of various sizes.}
\tempfigure{Assembly of rods in various solvents}

\subsection{Self-Assembly of Janus Rods}

\figthree{fig:janus-sizes}{figures/rods/janus-rods-4x2um.png}{figures/rods/janus-rods-6x2um.png}{figures/rods/janus-rod-cluster-zoom.png}{Various sizes of Janus rods.}

\figone{fig:assembly-v-solvent}{figures/rods/janus-rod-assembly-vs-solvent.png}{\linewidth}{
Self-assembly of Janus rods in (a) water, (b) DMSO and (c) IPA.}

\figone{fig:assembly-small-clusters}{figures/rods/janus-rods-large-clusters-ordered.png}{\linewidth}{
Janus rods self-assemble at low concentrations into small ``micellar'' structures.}

\figone{fig:large-struct}{figures/rods/janus-rods-large-structures.png}{\linewidth}{
Janus rods self-assemble at high concentration into extended, gel-like structures in water.}

\figone{fig:dilute-large-area}{figures/rods/janus-tiled-large-area.jpg}{\linewidth}{
A large-area view of Janus rod self-assembly in DMSO/water at a low concentration.}

\figone{fig:high-conc-large-area}{figures/rods/high-conc-large-area.png}{\linewidth}{
A large-area view of Janus rod self-assembly in DMSO/water at a high concentration}

\figone{fig:silver-embedded}{figures/rods/silver-fluorescence-assembly.png}{\linewidth}{
Ag nanoparticles are embedded in at 5 wt\% in otherwise-standard ETMPTA-20 monomer solution, and Janus rods
are fabricated as a demonstration of self-assembled functional Janus rods.}

\figone{fig:assembly-sim}{figures/rods/schweizer-tripathy-janus-rod-simulation.png}{\linewidth}{
Simulation of Janus rod assembly.}

\begin{itemize}
\done{Fabrication of Janus rods: various sizes (figure)}
\done{Comparison of self-assembly in various solvents (figure)}
\done{Small clusters vs large structures}
\done{Alignment of assembled rods}
\done{Image segmentation for analyzing structures}
\notdone{Some more good-looking images may be required}
\end{itemize}
