\documentclass[draftthesis,edeposit]{uiucthesis2009}
%\documentclass[fancy,fullpage]{uiucthesis2009}
\usepackage[usenames,dvipsnames]{color}

\usepackage{graphicx}

\newcommand{\microns}{$\mu m$}
\newcommand{\degC}{~$^{\circ}C$}

\newcommand{\done}[1]{
\item{\textcolor{Green}{#1}}
}

\newcommand{\notdone}[1]{
\item{\textcolor{BrickRed}{#1}}
}

\newcommand{\itemheader}[1]{
\item{\textbf{#1}}
}

\newcommand{\statement}[1]{
\item{#1}
}

\newcommand{\tempfigure}[1]{

\begin{figure}[h]
\caption{#1}
\end{figure}
}

\begin{document}

\msthesis
\department{Materials Science and Engineering}
\advisor{Jennifer Lewis}
\title{Fabrication, self-assembly and dynamics of anisotropic colloids}
\author{Adam DeConinck}
\schools{B.S. Physics, Michigan Technological University, 2007}
\committee{Professor Jennifer A. Lewis}

\maketitle

\frontmatter

\begin{abstract}
This work details the development of techniques to fabricate and study structured colloidal particles...
\end{abstract}

\begin{dedication}
To Leigh.
\end{dedication}

\documentclass[11pt]{article}
\usepackage{geometry}
\usepackage[T1]{fontenc}
\usepackage[usenames,dvipsnames]{color}

\pagestyle{empty}
\geometry{letterpaper,tmargin=0.75in,bmargin=0.75in,lmargin=1in,rmargin=1in,headheight=0in,headsep=0in,footskip=0in}

\setlength{\parindent}{0in}
\setlength{\parskip}{0.1in}
\setlength{\itemsep}{-0.2cm}
\setlength{\topsep}{0in}
\setlength{\tabcolsep}{0in}

\newcommand{\bigsection}[1]{	
	\vspace{4pt}
	{\fontfamily{phv}\selectfont\Large#1}

%	\vspace{-10pt} \rule{\textwidth}{1pt}
}

\newcommand{\microns}{$\mu m$}

\begin{document}

\fontfamily{ppl}\selectfont

\bigsection{Responsibilites to research group before I leave: \textbf{\textit{Progress So Far}}}

\textbf{Tasks in progress}
\begin{itemize}

\item \textbf{Masters Thesis:} Determined structure of thesis, laid out table of contents. Detailed notes on contents completed; 
see attached.

\item \textbf{Silica samples for new postdoc:} Liz + Ekandrea.

\item \textbf{My part of Willie's paper:} Done.
%: recheck all the math and code for the diffusion analysis, recheck text in paper.

\item \textbf{Nuzzo collaboration:} Scott is working on new draft for resubmission. I'm keeping involved in this as much as I can.

\item \textbf{Confocal microscope repair:} After additional conversations with Steve at Visitech, have determined that the piezo 
controller is at fault.  Visitech does not manufacture this component, but instead obtains it from Physik Instruments. Visitech is
handling the repair of the controller by PI on their end.

\end{itemize}

\textbf{Knowledge transfer}
\begin{itemize}

\item \textbf{Documentation of 81ESB fluorescence:} Partially done.

\item \textbf{SFL training + docs:} Good progress on training Steve. (Have not yet begun with John due to scheduling.) Docs not yet
written, but plan to use combination of my notes + Steve's notes to produce these.

\item \textbf{Confocal training + docs:} Have been waiting on confocal repairs; may have to begin training sooner. Group jobs have
assigned this to Steve. Jaci docs exist but still must be updated.

\item \textbf{Zetasizer:} Scheduled training for maintenance with Brett for next week.

\item \textbf{Wiki:} Transferred admin access + training to Steve.

\item \textbf{Particle tracking software:} Organized program files, not yet written docs. This for Gao?

\item \textbf{Train for clean-room master fabrication:} Waiting on Steve and John's cleanroom authorization.

\end{itemize}

\textbf{Lab maintenance}
\begin{itemize}

\item \textbf{SFL microscope in top condition:} Have ordered new supplies for microscope + SFL. Scheduling Olympus service visit with John.

\item \textbf{Confocal microscope in top condition:} Have ordered new supplies. Waiting on piezo.

\item \textbf{Clean and re-organize my lab space:} Not yet done.

\item \textbf{Disposition of chemicals:} Transferred essentials for SFL, fluorescent dyes, etc. to Steve. Misc chemicals to be collected in
waste pickup.

\end{itemize}

\pagebreak

\bigsection{Explanation of Thesis Outline}

The attached thesis outline is based on the Physics Dept LaTeX template for a UofI graduate thesis. I have
attached a proposed table of contents, and an outline of each chapter which contains sections and short 
bullet-point summaries.  The bullet points represent a ``first best guess'' at the contents of each section:
ideally, I should be able to go through this document and replace every one-three bullets with a paragraph.
Figures I plan to slot into each chapter are also listed.

\textbf{The bullet points are color-coded:}\\
\textcolor{BrickRed}{Red text represents incomplete experimental or analytic work.}\\
\textcolor{Green}{Green text represents experiments or analysis which are already complete.}\\
Black text represents background or explanatory text to be inserted.

I currently plan to begin writing with the sections for which there is no incomplete labwork. I will start with the
introduction and literature review, followed by the chapters on ``exotic'' particles and tracking algorithms for which
no new experimental work is planned. I have the chapter on rods placed later in my schedule (late September) so that
I can ensure no new experimental work is required first.  The appendices are last given their lower priority.

\end{document}

\tableofcontents
\listoffigures

\mainmatter

\chapter{Introduction}

\begin{figure}
\begin{center}
\includegraphics[height=1.5in]{figures/literature-review/sphere-crystal.png}
\includegraphics[height=1.5in]{figures/literature-review/sphere-gel.png}
\end{center}
\caption{Spherical colloids with purely repulsive interactions assemble into (a) ordered 
crystal structures with fcc geometry, while spheres with
attractive interactions may assemble into (b) open ``gel'' structures.}

\label{fig:isotropic-structs}

\end{figure}

As a class of materials, colloidal suspensions are of interest both
in the study of self-assembly~\cite{glotzer-solomon} in applications such
as photonic crystals~\cite{vos-photonic, yang-photonic} and three-dimensional templates for tissue 
engineering scaffolds~\cite{zhang-tissue}. However, despite the wide interest in these materials, the range of possible 
structures made available by the 
self-assembly of isotropically-interacting spherical colloids is relatively narrow.  Due to the isotropic nature
of colloidal interactions, there are only two assembled structures available.
When the interparticle interaction is purely 
repulsive, such as in a hard-core interaction,  a stable, ordered face-centered-cubic crystal
structure(Fig.~\ref{fig:isotropic-structs}(a)) with a volume fraction of 0.74 is formed.~\cite{ise-crystal, wong-crystal}
When the interparticle interaction is attractive, the particles may form an open, disordered ``gel'' 
structure (Fig~\ref{fig:isotropic-structs}(b)) 
with an essentially random arrangement and gap volumes which are potentially larger than the 
particle size.~\cite{warren-gel}

Many applications, such photonic crystals, would benefit from the availability of different
kinds of self-assembled structures.~\cite{glotzer-solomon}  One way to address this need is to 
introduce colloidal particles that incorporate one or more forms of anisotropy, in which the 
particle is altered such that the interaction between two or more particles becomes non-uniform depending on 
their relative orientations.  These alterations may be based on the shape of the particles, the chemical makeup
of the particles, or 
some combination of the two.

\begin{figure}[h]
\begin{center}
\includegraphics{figures/glotzer-anisotropy-dimensions.png}
\end{center}
\caption{Anisotropy dimensions proposed by Glotzer and Solomon~\cite{glotzer-solomon} to classify 
different forms of particle anisotropy.}
\label{fig:glotzer-dimensions}
\end{figure}


% Find many of the references in the following paragraph in Glotzer and Solomon's 2007 NatMat paper.
In a 2007 article in \textit{Nature Materials}~\cite{glotzer-solomon}, Glotzer and Solomon propose the system of anisotropy 
classification shown in Fig.~\ref{fig:glotzer-dimensions}. These 
anisotropy dimensions include shape-based dimensions such as aspect 
ratio, faceting, branching, shape gradient, 
and roughness (Fig.~\ref{fig:glotzer-dimensions}(B,C,E,G,H)), as well as dimensions based on the presence of
multiple chemistries such as surface coverage, pattern quantization, and chemical ordering 
(Fig.~\ref{fig:glotzer-dimensions}(A,D,G)).  These dimensions do not necessarily 
represent an exhaustive classification of the types
of anisotropy which are theoretically possible, but 
instead generalize from anisotropy types which have been observed in the
recent literature.  For example, rod-shaped or ellipsoidal particles of moderate aspect ratio have been fabricated 
by a wide variety of techniques including 
lithography~\cite{desimone-shear} and the stretching of colloidal spheres~\cite{rods-mohraz}; branched 
tetrapods have been fabricated of gold~\cite{gold-tetrapods},
and CdTe~\cite{cdte-tetrapods}; and chemically patterned particles have been produced
through microfluidic means~\cite{shepherd-janus}
as well as by conventional photolithography.~\cite{desimone-janus}  This list of dimensions may therefore
be seen as a useful framework for classification: by combining multiple dimensions, more complex types of particles may be 
designed (Fig.~\ref{fig:dimensions-combined}), or a complex particle may be classified in terms of which dimensions it includes.
New forms of anisotropy may be identified as those which cannot be decomposed into dimensions already identified.

\begin{figure}[h]
\begin{center}
\includegraphics{figures/glotzer-combine-dimensions.png}
\end{center}
\caption{Multiple anisotropy dimensions may be combined to yield more complex forms.~\cite{glotzer-solomon}}
\label{fig:dimensions-combined}
\end{figure}

In this work, we develop techniques for
the fabrication of colloids with geometric and chemical anisotropy and begin to characterize the dynamical behavior
and self-assembly of these particles.

\section{Thesis Scope}

The aim of this work is to develop techniques for the fabrication and characterization of anisotropic colloids and to begin
to explore their dynamical and self-assembly behavior.  Fabrication is based on flow lithography
techniques for producing polymeric particles~\cite{dendukuri-cfl, dendukuri-sfl}, 
and characterization is primarily based on fluorescence and confocal 
microscopy~\cite{weitz-confocal} and particle 
tracking.~\cite{crocker-grier-spheres,rods-mohraz}
The systems used are based on a combination of a hydrophobic monomer (tri(methylol propane) triacrylate) and hydrophilic 
monomers (poly(ethylene glycol) diacrylate and 20-mol ethoxylated tri(methylol propane) triacrylate). 
Single-component particles are used to study the effects of
geometry on dynamical behavior in isolation, while multiple-component particles introduce 
the hydrophobic interaction to induce self-assembly.  Particles are suspended in a 
variety of solvents to explore this interaction, including
water, ethanol, dimethyl sulfoxide, isopropanol, and toluene.

\section{Thesis Organization}

Chapter~\ref{ch:comp-tracking}
details algorithms and software developed in the course of this study to analyze microscopy images containing 
anisotropic colloids.  Chapter~\ref{ch:rods} investigates the fabrication, behavior and self-assembly of simple rod-shaped colloids
in both single-component and ``Janus'' forms, while Chapter~\ref{ch:exotic} 
investigates colloids with more exotic geometries. The main
conclusions are presented in Chapter~\ref{ch:conclusions}.  
Appendix~\ref{sec:matlab-implementation} details the Matlab implementation of the algorithms developed in
Chapter~\ref{ch:comp-tracking}.


\include{literature-review}

\chapter{Computerized Tracking of Anisotropic Colloids}
\section{Introduction}

Confocal laser scanning microscopy (CLSM) is a powerful technique for the study of three-dimensional
structure in fluorescent materials. When applied to fluorescent colloids, CLSM enables the observation
and identification of individual particles, determining their positions in three-dimensional space.
These particle locations alone can be used to derive a great deal
of information about the material, such as the distribution of number of nearest neighbors and
the pair distribution function (PDF). Repeated observations at regular intervals allow for dynamical
measurements of parameters such as the diffusion constant, and may be used to study the microstructural
differences between different parts of the colloidal phase diagram. CLSM has the additional advantage
that since it produces real-space position data, it has substantial advantages over scattering techniques
in coping with samples with highly asymmetric structures.~\ref{?}

However, 
the production of 3D structural information requires more than just a powerful imaging technique: it also
requires powerful computational analysis to translate image data into a list of particles and positions, and
to determine the relevant physical data from this list.  In addition, the behavior of non-spherical colloids
is governed not only by the relative positions of the particles but also their orientations. Developing an
understanding of anisotropic colloids
therefore calls for the development of image processing techniques for the extraction and analysis of
structural data from microscopy images.

\section{Literature review}

The identification and tracking of single particles is a powerful tool for the characterization 
of colloidal materials.  

\subsection{Tracking of spherical colloids}

In a 1996 paper in the \textit{Journal of Colloid and Interface Science}~\cite{crocker-grier-spheres},
John Crocker and David Grier outline a five-stage procedure for the tracking of
spherical colloidal particles.  This algorithm was implemented in IDL and has become the standard for carrying out 
3D tracking of spherical particles, and has been ported to other environments such as 
Matlab~\cite{blair-dufrense-matlab, kilfoil-matlab} and LabView~\cite{optical-trapping-group} which are
also commonly used in the scientific community.

\subsubsection{Image restoration}

The digital confocal imaging of 
fluorescent spherical colloids can introduce introduce geometric distortions and single-pixel
noise into the image.  This can be compensated for by applying a band-pass filter 
to the image $A$, which is composed of two parts.
The first part is a boxcar average over a region of extent $2w + 1$, where $w$ is an integer
larger than a single sphere's apparent radius in pixels, but smaller than an intersphere separation:

\begin{center}\begin{equation}A_w(x,y) = \frac{1}{(2w+1)^2} \sum_{i,j=-w}^w A(x+i,y+j)
\end{equation}\end{center}

The second part is a convolution of the image with a Gaussian surface of revolution with 
half-width $\gamma_n \approx $ 1 pixel:

\begin{center}\begin{equation}A_{\gamma_n}(x,y) = \frac{1}{B} \sum_{i,j=-w}^w A(x+i,y+j)\exp{\left(-\frac{i^2+j^2}{4\gamma_n^2}\right)}
\end{equation}\end{center}

with normalization B = $[\sum_{i=-w}^w \exp{-(i^2/4\gamma_n^2)}]^2$.

These two filters can be applied simultaneously in a single step using the convolution kernel

\begin{center}\begin{equation}K(i,j) = \frac{1}{K_0} \left[ \frac{1}{B} \exp{ \left( -\frac{i^2+j^2}{4\gamma_n^2} \right)} -
\frac{1}{(2w+1)^2} \right]
\end{equation}\end{center}

The normalization constant $K_0 = 1/B[\sum_{i=-w}^w \exp{-(i^2/2\gamma_n^2)}] - (B/(2w+1)^2)$ facilitates comparison
among images filtered with different values of $w$.

\subsubsection{Locating particles}

Particles are located by identifying local brightess maxima within an image. A pixel is identified
as a candidate if no other pixel within a distance $w$ is brighter, where $w$ is the same value used in
the filtering step. This was implemented by Crocker and Grier using a gray-scale dilation operation.

\subsubsection{Refining location estimates}

Having located a brightness maximum at $(x, y)$ which is presumably near
a sphere's geometric center at $(x_0, y_0)$, additional refinements are possible
which may achieve sub-pixel accuracy.  An offset $(\epsilon_x, \epsilon_y)$ is
calculated according to:

\begin{center}
\begin{equation}
\left( \begin{array}{c} \epsilon_x \\ \epsilon_y \end{array} \right) 
= \frac{1}{m_0}
\sum_{i^2+j^2 \leq w^2} 
\left( \begin{array}{c} i \\ j \end{array} \right)
A(x+i,y+j)
\end{equation}
\end{center}

Here, $m_0 = \sum_{i^2+y^2 \leq w^2} A(x+i,y+j)$ is the integrated brightness of the
sphere's image. The refined location estimate is then $(x_0, y_0) = (x+\epsilon_x, y+\epsilon_y)$.
If either $|\epsilon_x|$ or $|\epsilon_y|$ exceeds 0.5, the candidate centroid location can be moved and the
refinement recalculated.

\subsubsection{Noise discrimination and tracking in depth}

During the centroid refinement calculations, two moments of each sphere image's brightness 
distribution are calculated.  The first is $m_0$, and the second is 

\begin{center}\begin{equation}m_2 = \frac{1}{m_0} \sum_{i^2+j^2 \leq w^2} (i^2 + j^2)A(x+i,y+j)
\end{equation}\end{center}

where $(x,y)$ are the refined centroid locations.  The distribution of the $(m_0,m_2)$ data reflects the
sphere's positions along the direction normal to the imaging plane, and a control experiment using a 
monolayer of particles is used to calibrate this data.

\subsubsection{Linking locations into trajectories}

Having located colloidal particles in a sequence of video iages, it is possible to 
match locations in each image with corresponding locations in later images to produce
trajectories.  This requires determining which particle in a given image
most likely corresponds to one in the preceding image.  The tracking of 
multiple particles requires that we seek the most probable set of $N$ identifications 
between $N$ locations in two consecutive images. If the particles are indistinguishable (as for
monodisperse colloidal particles), this likelihood can be estimated only using relative
proximity.

The probability that a single Brownian particle will diffuse a distance $\delta$ in the plane 
in time $\tau$ is

\begin{center}\begin{equation}P(\delta|\tau) = \frac{1}{4\pi D\tau} \exp{ \left( -\frac{\delta^2}{4D\tau} \right) }
\end{equation}\end{center}

where $D$ is the particle's self-diffusion coefficient.  For an ensemble of $N$ noninteracting
identical particles, the corresponding probability distribution is the product of the 
single-particle results:

\begin{center}\begin{equation}P({\delta_i}|\tau) = \left( \frac{1}{4\pi D\tau} \right)^N 
\exp{ \left( -\sum_{i=1}^N \frac{\delta_i^2}{4D\tau} \right) }
\end{equation}\end{center}

Each label assignment can be thought of as a bond drawn between a pair of particles
in consecutive frames.  $P({\delta_i}|\tau)$ is calculated for all possible combinations which
represent a displacement below some characteristic length scale $L$, selected by the user based on
the experimental conditions.

\subsection{Rod tracking}

While the algorithm by Crocker and Grier is used widely for tracking spherical particles, it 
cannot deal with particles which have an anisotropic shape and some orientation. 
To begin to deal with simple anisotropy, Mohraz and Solomon have developed an algorithm for tracking 
ellipsoidal colloidal rods based on the spherical tracking algorithm.~\cite{rods-mohraz, solomon-dynamics}

\figone{fig:pmma-fabrication}{figures/computerized-tracking/pmma-fabrication.png}{0.6\linewidth}{
(a) Synthesis of PMMA-g-PDMS spheres, (b) curing of the PDMS matrix, (c) uniaxial deformation,
(d) rod harvesting.}

Poly(methyl methacrylate)-g-poly(dimethylsiloxane) (PMMA-g-PDMS) fluorescent colloidal spheres were
synthesized and suspended in a polymerizable liquid silanol-terminated PDMS, which was then cured to form a solid 
matrix.  This matrix was then heated above the glass transition temperature and subjected to 
uniaxial stretching, then cooled while still deformed.  The rods were then harvested from the 
elastic film by chemical degredation, and transferred to a cyclohexyl bromide (CXB) solution.
This rod suspension was then visualized via confocal laser scanning microscopy, and 
subjected to a three-stage image processing algorithm to determine the position and orientation of each rod.

\subsubsection{Image restoration}

To correct for imaging distortions and local noise, voxels are convoluted with neighbors found within a local
distance $w$ using a Gaussian function, where $w$ is of the order of the rod half-width. The resulting
voxel intensity $A(x,y,z)$ is

\begin{center}
\begin{equation}
A(x,y,z) = \frac{1}{B(x,y,z)} \sum_{i,k,j=-w}^w A(x+i,y+j,z+k) 
\exp{ \left( -\frac{i^2+j^2+k^2}{6\lambda^2} \right)}
\end{equation}
\end{center}

where $B$ is a normalization constant and $\lambda$ is defined to be 1 for these experiments.

\subsubsection{Rod backbone identification}

\figone{fig:local-line-max}{figures/computerized-tracking/local-line-max.png}{0.6\linewidth}{
Local line maximum criterion.}

Rod backbones are identified using a local line maximum criterion.  Each voxel is compared with its immediate 
neighbors in all directions along lines with length $2w+1$.  If a candidate voxel is found to be the brightest
point on more than a critical fraction of these lines (typically 70\%), it is considered to be a backbone
pixels.  Backbone pixels are then grouped together by cluster analysis to form rod backbones.

\figone{fig:pmma-rod-backbones}{figures/computerized-tracking/backbone-assignment.png}{\linewidth}{
Rod backbone assignments (a) in a stretched film and (b) in a sediment structure.}

\subsubsection{Orientation and centroid calculation}
\label{sec:orient-calculate}

\figone{fig:coordinates}{figures/computerized-tracking/rod-schematic-solomon.png}{0.5\linewidth}{
Positional and orientational coordinates for colloidal rods.~\cite{mohraz-rods}}

Once the backbone pixels for a given rod have been identified, the position and orientation of each rod may
be determined based on these pixels' locations.  Each rod's geometric configuration can be
completely specified by three positional coordinates, $x$, $y$ and $z$, and two
orientational coordinates $\theta$ and $\phi$, as shown in~\ref{fig:coordinates}.

The rod's center-of-mass position may be calculated
straightforwardly based on a simple average over the positions of the backbone pixels.  In 
\ref{eq:x-cent}--\ref{z-cent}, $r_{0,i}$ represents the center-of-mass of coordinate $r$ for 
rod $i$.  $S_i$ is the set of identified backbone pixels associated with that rod, and $s$ is the index
variable summing over that set.

\begin{equation}
\label{eq:x-cent}
x_{0,i} = \frac{1}{S_i} \sum_{s}^{S_i} x_{s,i}
\end{equation}
\begin{equation}
\label{eq:y-cent}
y_{0,i} = \frac{1}{S_i} \sum_{s}^{S_i} y_{s,i}
\end{equation}
\begin{equation}
\label{eq:z-cent}
z_{0,i} = \frac{1}{S_i} \sum_{s}^{S_i} z_{s,i} 
\end{equation}

Calculating orientation is somewhat more complex.  First, for each dimension $r$, the quantity
$|<l_r^2>^{1/2}|$ is calculated (\ref{eq:x-len}--\ref{z-len}). 
Geometrically, this is the size of the projection of the rod's length
onto the axis $r$, and it is equivalent to the standard deviation of the $r$ coordinate.
Once these dimensions have been calculated, the angles $\phi$ and $\theta$ may be 
determined according to \ref{eq:phi} and \ref{eq:theta}, respectively.

\begin{equation}
\label{eq:x-len}
|<l_x^2>_i^{1/2}| = \left[\frac{1}{S_i} \sum_s^{S_i} (x_{s,i} - x_{0,i} )^2 \right]^{1/2}
\end{equation}
\begin{equation}
\label{eq:y-len}
|<l_y^2>_i^{1/2}| = \left[\frac{1}{S_i} \sum_s^{S_i} (y_{s,i} - y_{0,i} )^2 \right]^{1/2}
\end{equation}
\begin{equation}
\label{eq:z-len}
|<l_z^2>_i^{1/2}| = \left[\frac{1}{S_i} \sum_s^{S_i} (z_{s,i} - z_{0,i} )^2 \right]^{1/2}
\end{equation}

\begin{equation}
\label{eq:theta}
\theta_i = \cos^{-1} \left(\frac{<l_z^2>_i^{1/2}}{<l^2>_i^{1/2}} \right)
\end{equation}
\begin{equation}
\label{eq:phi}
\phi_i = \tan^{-1} \left(\frac{<l_y^2>_i^{1/2}}{<l_x^2>_i^{1/2}} \right)
\end{equation}

\section{Algorithm for tracking SFL rods}
\label{sec:rod-tracking}

Our method for locating and tracking rods produced by stop-flow lithography (SFL) draws heavily from that published by 
Mohraz and Solomon for tracking PMMA rods.~\cite{rods-mohraz}.
While this algorithm performs very well for a restricted class of rods, it fails in cases where the particle
cross-section is not circular, and points near the particle backbone are not guaranteed to produce higher 
intensities than their immediate neighbors.  This is the case for our ``rods'' produced by stop-flow
lithography (SFL), in which the sides of the rods are relatively flat due to the fabrication
geometry. These particles have correspondingly flat fluorescence profiles, and require a more complex analysis
to calculate a ``backbone''.

We have developed an algorithm for processing 2D and 3D CLSM data of fluorescent SFL rods to
produce position and orientation data.  Starting from raw CLSM images, this algorithm can be divided
into several phases, including (i) image cleanup; (ii) segmentation; (iii) skeletonization;
(iv) position calculation; and
(v) particle tracking over the time series.  

A note on terminology: the algorithm described below is identical for both 2D and 3D images, as all
operations are defined for both cases and used identically. However, where the individual elements of
2D images are referred to as pixels, the elements of 3D images are generally referred to as voxels.
For simplicity, all such elements are referred to as pixels in the explanation below.

\subsection{Image cleanup}

Two different image cleanup methods were considered, and used depending on their effectiveness with
sample data.

The first method, a real-space bandpass filter, is derived from the filter published in the 
Matlab implementation of spherical particle tracking~\ref{crocker-weeks} by Blair and
Dufrense. This filter performs a band-pass by convolving the image with two kernels: 
a Gaussian kernel and a boxcar kernel.  The Gaussian convolution performs the low-pass
operation, while subtracting the result of the boxcar convolution from the Gaussian result
performs the high-pass operation. This filter takes two parameters, the characteristic scale
of image noise (generally equal to one pixel) and the typical particle size.  This works well,
but has issues in images with multi-pixel noise.

While the band-pass performed well on many images, some experiments produced data with noise or
extraneous features which did
not easily yield to the bandpass operation. This can be attributed to the fact that SFL fabrication produces
solutions which have some amount of fluorescent monomer present in the liquid as well as the particles, which
could not always be removed effectively.  A second method was devised using morphological
operations to better neutralize non-particle features.

This method may be divided into five steps. First, the 
image is run through a morphological top-hat transform. This transform is used to account for the effects of
uneven illumination in the image. Then the image is thresholded to produce a binary image, where the 
background is black and the fluorescent features are white. The threshold is selected such that pixels which 
are part of the particle volume are never assigned to the background; Otsu's criterion was found to be reliable
for this.~\ref{?}  Next, a binary opening is applied with an isotropic structuring element to
suppress small features. The size of the structuring element is selected manually by the user, but a 
reliable choice was found to be a diameter roughly equal to half the width of the typical rod. 

At this point a binary image has been produced which suppresses most non-particle features, but morphological
image operations are not guaranteed to preserve shape and orientation of image features.  To retain the noise suppression
but regain the original shape, we perform one additional morphological dilation using the same structuring element to
guarantee that the foreground regions fully overlap with the rods, then perform a binary AND between the result and the
original image. This is effectively equivalent to using the result of our morphological operations as a mask on
the original, suppressing
all pixels which are marked as background.

\subsection{Segmentation}

The next step of the algorithm is image segmentation, in which individual particles are identified and each pixel
in the image is assigned to a specific particle, or to the background. This is especially important in the study 
of Janus rods, which are come into contact during self-assembly and which therefore often touch or overlap in 
CLSM images. 

First, the image is thresholded to produce a binary image, with a threshold selected such that all pixels which
are part of the rods are assigned to the foreground. This may generally be accomplished through the use of 
Otsu's criterion.~\ref{?}

Second, the distance transform is calculated.  In this step, each foreground pixel is assigned a number which
gives the distance between this pixel and the nearest background pixel. In this algorithm, the distance measure used
is simple Euclidean distance, calculated center-to-center between this pixel and the closest background pixel.  Alternative
distance measures such as the ``chessboard'' measure may also be used to speed up computation, but these measures were
found to negatively impact segmentation.  To prepare the image for watershed segmentation, the distance 
transform is transformed such that all distances are made
negative, while the background remains at a flat zero. An h-minima transform is applied to remove small local minima
due to noise in the image.

The primary segmentation step is the watershed transform. In this transform, a gray-level image is viewed as
a topographic relief map where the pixel values represent altitude. A drop of water falling on a relief 
surface will run down to a minimum, and many drops will fill any basins present until the basins merge.
Implementations of the watershed transform use this concept to calculate the boundaries between catchment basins,
fully segmenting the image.  The number of basins is calculated either by using local minima in the image, or by 
pre-assigning a set of markers.  In our algorithm, we generally use the Matlab implementation of \texttt{watershed} which
uses the local minima method.  This carries some danger of over-segmentation (mitigated somewhat by the 
h-minima transform), but is better suited to automatic processing of a large number of images than manual markers.

\texttt{watershed} outputs an image which labels each pixel according to a region ID number, and labels both background
and foreground pixels with these regions. To restrict these labels so that the background is labeled separately, all 
watershed pixels which correspond to background-valued pixels in the thresholded image are assigned a label number of zero.

\subsection{Skeletonization}

Once we have identified which image pixels belong to each particle, we need to put this data into a form from
which reliable position and orientation information can be calculated.  While it is tempting to simply
calculate the centroid and associated moments from the raw pixel data, this can be problematic when working with
time-series data due to boundary noise. Consider a single foreground pixel belonging to 
an identified particle which is experimentally
constrained to be stationary, and which is adjacent to a background pixel because it is on the edge of the identified 
region. In the next image in the time series, this pixel's intensity is reduced and it is identified as a background
pixel.  If we are calculated particle position as the average of all the identified pixels, this will result in 
the calculated position changing, even if the particle did not physically move. The next frame after that, it 
may be re-identified as a foreground pixel.  While the effect is small, experimental conditions may magnify
these effects and produce appreciable fluctuations in the position and orientation. One way to get around this issue
is to calculate a particle ``skeleton'' which is less sensitive to this form of noise.

For a rod, we calculate a backbone very similar to the backbone calculated in the Mohraz-Solomon algorithm; but rather 
than using an intensity gradient, we instead calculate with respect to the particle geometry. For each particle, we 
isolate its pixels from the environment (i.e., generate a new image containing only this particle). We then
calculate the distance transform to allow identification of particles which are closest to the 
geometric backbone.  These are identified by applying a ``percentile threshold'' in which the all pixels which
fall below a given percentile in the distribution are reassigned as background pixels. Typical cutoffs used
are 90-95\%, depending on trial images.

\subsection{Calculation of position and orientation}

Position and orientation are calculated in an identical fashion as in Mohraz and Solomon~\cite{mohraz-rods}; see
Section~\ref{sec:orient-calculate} for details.

\section{Implementation}
\label{sec:matlab-implementation}

The results of typical experiments with samples of single-component or Janus rods included both 
large-area tiled images taken of many particles at a single time,
used to study static self-assembly, and movies consisting of many subsequent images in a 
particular location, taken to study dynamical behavior.  Each of these experimental types required the processing
of image sets numbering in the hundreds or thousands.  The analysis of these images using the algorithm developed
above requires the selection of a number of input parameters, such as the thresholding levels and the 
structuring elements for morphological processing.  However, while all the images from a typical
experiment could be expected to share the same parameter values, those values might vary considerably for the 
analysis of different
experiments.

To address this, we implemented our algorithm as a set of independent functions which could be carried out manually 
or called from an automated script. A typical analysis was carried out by selecting one or more 
test images from the data set; carrying out the various image processing steps on these test images, varying
the processing parameters to obtain the best results; and then calling the automation script using the 
optimized values to process the entire data set.  Analysis of test images was generally carried out on a single
workstation, while full-dataset processing was carried out on dedicated servers to maximize processing efficiency.

All steps of the analysis were implementing in Matlab~\cite{matlab}, making heavy use of functions from the
Image Processing Toolbox.  The following description summarizes the process of carrying out a manual analysis of
test images as a guide to future users; a full source-code listing can be found in~\ref{sec:matlab-code}.

\begin{lstlisting}[label=ls:manual,caption=Typical sequence of a manual analysis]
% Load the image.
img = imread('filename.tif');

% If using bandpass cleaning.
clean = bpass(img,lnoise,lobject);

% If using morphological cleaning.
%  top-hat step: radius 50 is greater than rod size.
%  opening step: radius 2 is a good size for eliminating small variations.
thstruct = strel('disk',50);
opstruct = strel('disk',2);

% Segmentation step.
watershed_labels = segment(clean, 1);

% Now find backbone pixels.
skeletons = backbones(clean, watershed_labels, 90);

% Finally get the list of positions.
positions = calc_positions(skeletons);
\end{lstlisting}

\subsection{Image cleanup}

Image cleanup is probably the part of the analysis which is most sensitive to parameter selection.
Original images of colloidal rod samples from CLSM or FM are often noisy or unevenly illuminated, and these
effects vary from experiment to experiment.  However, subsequent steps of the analysis assume that their input
images will be simple binary images, with white rods and a black background. Choosing the correct parameters 
to produce such images is a matter of trial and error, and the particular choices must be re-optimized
for each new experiment.  Two options exist for performing this clean-up: a simple bandpass filter, 
and a more complex set of morphological operations.

\subsubsection{Option 1: band-pass}

\texttt{clean = bpass(image\_array, lnoise, lobject[, threshold])}

\begin{itemize}
\item \texttt{image\_array}: Matlab array containing image pixels.
\item \texttt{lnoise}: Characteristic length-scale of noise.
\item \texttt{lobject}: Characteristic length-scale of object to be tracked (i.e., length of a colloidal rod).
\item \texttt{threshold}: By default, the last step of this algorithm is to set all negative pixels (generated
by the convolution) to be zeros. This parameter may optionally be used to apply a threshold with a different cut-off.
\end{itemize}

\texttt{bpass.m} is from the Blair and Dufrense~\cite{blair-matlab} Matlab implementation of the Crocker and Weeks
package for tracking of spherical particles.~\cite{crocker-tracking}

\subsubsection{Option 2: morphological cleanup}

\texttt{clean = mclean(img, thstruct, opstruct[, threshold])}

\begin{itemize}
\item \texttt{img}: Matlab array containing image pixels.
\item \texttt{thstruct}: Matlab structuring element used to carry out the top-hat transform.
\item \texttt{opstruct}: Matlab structuring element used to carry out the opening and dilation operations.
\item \texttt{threshold}: By default, the threshold used in this algorithm is selected using the built-in
Matlab function \texttt{graythresh}, which uses Otsu's algorithm~\cite{?}. Here, the user may optionally
select a different threshold.
\end{itemize}

\subsection{Segmentation}

\texttt{watershed\_img = segment(img[, height])}

\begin{itemize}
\item \texttt{img}: Matlab array containing the image pixels.
\item \texttt{height}: Maximum height to suppress in h-minima transform. Optional, defaults to 1.
\end{itemize}

\texttt{segment.m} carries out the image segmentation part of the algorithm, and consists of three calls to
Matlab built-in functions: \texttt{bwdist}, which carries out the distance transform, 
\texttt{imhmin}, which carries out the h-minima transform, and \texttt{watershed}, which performs watershed
segmentation. While these processes are all computationally intensive, the result is relatively insensitive to
processing parameters. The only parameter available is the height of the h-minima transform, which is generally
set to 1 to account for single-pixel fluctuations; 
it is increased only when over-segmentation is observed in the resulting watershed.~\ref{fig:over-segment}
To observe the resulting watershed segmentation, use the code in Listing~\ref{ls:showwater}.

\begin{lstlisting}[label=ls:showwater,caption=Observe watershed segmentation]
imshow( label2rgb(watershed_img,'jet') );
\end{lstlisting}

\subsection{Skeletonization}

\texttt{skeletons = backbones(img, watershed\_img, percent)}

\begin{itemize}
\item \texttt{img}: Matlab array containing the image pixels.
\item \texttt{watershed\_img}: Image containing the watershed labels.
\item \texttt{percent}: Percentile used in the thresholding step.
\end{itemize}

The generation of the rod skeletons, carried out in \texttt{backbones.m}, is also a relatively simple
procedure.  The results of the watershed segmentation step are used to find the portion of the 
image which contains each rod, and this is used to generate a new image in which the rod may
be analyzed in isolation. The only choice which must be made is the percentile threshold for selecting 
backbone pixels. This is again a matter of trial and error, but typical values are in the range of 90-95\%.

A final image is generated which contains all the backbones, with the pixels belonging to each one having
the value of their watershed label. This allows them to be uniquely identified in the following step. The background is 
assigned again to zero.  Observation of these backbones for quality check requires a thresholding step. Observation
of all the backbones may be accomplished using the code in Listing~\ref{ls:allbb}, while observing only the backbone 
with label $n$ may be accomplished using the code in Listing~\ref{ls:onebb}.

\begin{lstlisting}[label=ls:allbb,caption=Show all backbones as an image]
imshow(im2bw(skeletons, 0));
\end{lstlisting}

\begin{lstlisting}[label=ls:onebb,caption=Show only backbone with label $n$]
temp=skeletons;
temp(temp~=n)=0;
imshow(im2bw(skeletons, 0));
\end{lstlisting}

\subsection{Coordinate calculation}

\texttt{positions = calc\_positions(skeletons[, cutoff])}

\begin{itemize}
\item \texttt{skeletons}: Image containing the rod skeletons.
\item \texttt{cutoff}: Optional parameter listing a cut-off for ignoring a backbone.
\end{itemize}

For each individual rod, \texttt{calc\_positions.m} calculates the positional and
orientational coordinates and saves them to the array \texttt{positions}.  
\texttt{cutoff} is an optional parameter which allows \texttt{calc\_positions.m} to 
ignore backbones which contain under a certain number of pixels, as one last noise-protection step.

\texttt{positions} is a 2D Matlab array in which each row represents one backbone, and has 
the structure:

\begin{tabular}{ | c | c | c | c | c | }
\hline 
$x$ & $y$ & $z$ & $\phi$ & $\theta$ \\
\hline
\end{tabular}

In any 2D image, $z$ and $\theta$ are always zero.

\subsection{Automated analysis of time series}

\texttt{poslist = rod\_tseries\_bp(imgstack,lnoise,lobject,threshold,height,percent)}

\texttt{poslist = rod\_tseries(imgstack,thstruct,opstruct,threshold,height,percent)}

\subsection{Temporal tracking}

\texttt{tracks = track(xyzs, maxdisp, param)}

\begin{itemize}
\item \texttt{xyzs}: An array containing a time-sorted list of particle positions (and, here, orientations).
\item \texttt{maxdisp}: Maximum allowed displacement of a particle between frames.
\item \texttt{param}: A data structure containing additional processing parameters.
\end{itemize}

Frame-to-frame tracking of unique rods to form trajectories was accomplished using a Matlab routine, \texttt{track.m}, 
supplied by Blair and Dufrense~\cite{blair-matlab}, implementing the standard algorithm for particle tracking by
Crocker and Weeks~\cite{crocker-tracking}.  \texttt{track.m} requires that the data be in a time-sorted format where each
row consists of a list of coordinates followed by a frame number, and the frame numbers increase monotonically.

The output, \texttt{tracks}, is in a similar format which includes one extra column: a particle ``id number'' which
allows each trajectory to be identified. Rows are ordered so that particle id number increases monotonically, and
time increases monotonically within each set of particle rows.

The optional input structure, \texttt{param}, contains a variety of settings which alter the behavior of 
\texttt{track}.  The only setting important to this analysis is \texttt{param.dim}, which tells the program how
many columns to use as positional dimensions. Any additional columns will be ignored in the tracking calculation and 
simply ``carried along'' when the new array is built; this gives us a place to put our orientation coordinates, which
will not be used in the tracking routine.

\subsection{Characterization of rod suspensions}

The location and tracking of colloidal rods within experimental images is not an end in itself, but the
first step in determining the characteristics of the suspensions they are used to measure.  Data on the 
position and orientation of all rods within a suspension is an extremely useful tool, and the values of many
dynamical or structural properties may be directly calculated or inferred from this information.
While the current work did not proceed so far as to complete a detailed study of all these properties, or 
implementations of the calculations necessary for such a study, some preliminary work has been done in this
area which is worth exploring.

\subsubsection{Dynamics}

%\texttt{traj\_vis(tracks[, t0, t1])}

\texttt{showmovie(imgstack,tracks)}

%\texttt{traj\_vis.m} is a simple diagnostic tool which plots the trajectories for all the particles. 
%\texttt{t0} and \texttt{t1} are optional beginning and ending times.  
\texttt{showmovie.m} visualizes the
movie from the image stack, plotting a dot-and-bar for each tracked particle on the image to indicate 
the tracked position and orientation.

\texttt{disps = msd(tracks)}

\texttt{avgs = avg\_msd(disps)}

\texttt{msd.m} calculates the mean-square displacement (MSD) for position and orientation for each particle 
in \texttt{tracks}, and returns a data structure \texttt{disps} which contains each individual MSD vs time 
series.  \texttt{avg\_msd.m} takes this structure and averages across particles to produce plots similar 
to Figure~\ref{fig:example-diffusion-results}. Note that this averaging takes place at each 
individual particle's \textit{own} frame 2, 3, etc. so that the displacement at frame 2 of two particles 
will be averaged together--even if the second particle did not appear until a later frame.  This reduces the
total time over which averaging may take place, but improves statistics.

\subsubsection{Structure}
\label{sec:structure-calcs}

\texttt{result = nearest\_neighbors(positions, size)}

\texttt{pdf = pair\_dist(positions)}

\texttt{result = orient\_corr(positions)}

\texttt{nearest\_neighbors.m} calculates the average number of nearest neighbors for the particles in 
\texttt{positions} within a distance given by \texttt{size}. \texttt{pair\_dist.m} produces the pair-distribution
function (PDF) for all particles in \texttt{positions}, where as \texttt{orient\_corr.m} shows how the 
dot-product between particle orientations changes with distance.

\section{Results and discussion}

\subsection{Dynamics}

\figone{fig:example-diffusion-results}{figures/computerized-tracking/rod-diffusion-results-old.jpg}{0.6\linewidth}{
Translational and rotational diffusion results for 12~\microns~SFL-fabricated rods.}

Poly(ethylene glycol) rods with dimensions 5 \by 3 \by 3 \microns~were fabricated via SFL, suspended in 
water, and their diffusion was observed as described in Section~\ref{sec:exp-diffusion} with an 
image collection
frame-rate of 2 fps.  The resulting image sequence was imported into Matlab, and tracking was carried out
using the scripts outlined in Section~\ref{sec:matlab-implementation}.  Noise reduction was carried out using 
a band-pass filter with noise and particle dimensions of 1 pixel and 30 pixels respectively. 
Watershed segmentation was carried out using a \texttt{height} parameter of 1, and skeletonization
was performed using a percentile filter of value 0.95 to produce backbones.  The maximum allowable 
displacement between subsequent frames, \texttt{maxdisp}, was set to be 15 pixels.  Finally, the 
MSD of translational and rotational diffusion were calculated and averaged to prodice the results in 
Figure~\ref{fig:example-diffusion-results}.

A series of experiments was carried out to study the effect of rod size and aspect ratio on 2D diffusion, as 
detailed later in Section~\ref{sec:rod-diffusion-results}.

\subsection{Structure}

\figone{fig:track-assembly}{figures/assembled-janus-tracked.png}{0.7\linewidth}{
Assembled Janus rods are tracked.}

Janus rods are fabricated with hydrophilic and hydrophobic components to induce self-assembly.
In Figure~\ref{fig:track-assembly}, the results of a 2D image analysis for aligned Janus clusters
are shown.  This cluster is relatively easy to track, given the highly aligned orientations of the
rods and the flatness of the clusters; a three-dimensional cluster with rods ``stacked'' would require
3D confocal imaging to achieve segmentation.

An series of experiments was carried out to study the self-assembly of Janus rods, varying
the solvent conditions for assembly and rod aspect ratio.  Orientational ordering and
number of nearest neighbors are calculated across this experiment series; the results are
detailed in Section~\ref{sec:assembly-janus-rods}.


\chapter{Assembly and Dynamics of Rod-Shaped Colloids}
\section{Introduction}

When one is first presented with the different dimensions of anisotropy proposed by 
Glotzer and Solomon (Figure~\ref{fig:glotzer-dimensions},\ref{glotzer-solomon-assembly}), the
potential variety of particle types can be overwhelming. It is therefore useful to begin by studying 
particles which draw from only one or two of these anisotropy dimensions. To this end, our
initial study has focused on particles which vary only in the dimension of 
aspect ratio, i.e. colloidal rods. This study covers the fabrication of both single-component and 
Janus rods by stop-flow lithography, the study of rod diffusion by particle tracking, and the basics
of self-assembly for Janus rods in different solvents.

%\tempfigure{Colloidal rod examples}
%\begin{itemize}
%\statement{Natural rod systems}
%\statement{Systems studied by Solomon}
%\statement{Janus colloids: Granick}
%\statement{Interest in Janus rods}
%\end{itemize}

\section{Experimental Procedure}

Colloidal rods were fabricated by stop-flow lithography (SFL) as described in section~\ref{sec:SFL} using
hydrophobic and hydrophilic monomer solutions.  

\subsection{Microchannel device fabrication}

\figone{fig:device-design}{figures/rods/four-channels-together.jpg}{0.8\linewidth}{
Four Y-channel microchannel patterns, sharing a common reservoir for accumulation of particles from multiple experiments.}

Particle fabrication was carried out in Y-junction microchannel devices (Figure~\ref{fig:device-design}).
The primary channel of these devices had a typical 
height of 7 \microns, width of 200 \microns, and length of 3000-5000 \microns. These dimensions were selected
to facilitate the fabrication of large numbers of particles with small size in all dimensions: the low height
facilitated small-particle fabrication by limiting the height of the particles, while the comparatively large
width allowed many particles to be fabricated simultaneously. Multiple entrances were defined to allow up to three
monomer streams to be simultaneously flowed, with a single exit point for collecting particles.

\subsubsection{Photoresist masters}
Positive-relief photoresist master templates were fabricated by UV photolithography. A thin film
of SU-8 2007 photoresist (Microchem) was laid down on a clean Si wafer via spin-coating at 3000 rpm to produce a 
7 \microns layer. Next, a ``soft bake'' was carried out by heating the wafer on a hot plate at 120\degC
for five minutes to evaporate the photoresist solvent.  The device features were patterned by exposing the 
photoresist to UV light ??? for 40 s through a photomask defining the device design. A ``hard bake'' step
was then carried out by heating the wafer at 120\degC for ten minutes, to cure the photoresist in the exposed areas.
Finally, the wafer was immersed in SU-8 developer (Microchem) and agitated for two minutes to remove the unexposed
photoresist, then rinsed with isopropyl alcohol (IPA).

After fabrication, photoresist masters were subjected to a fluorinated silane vapor coating to inhibit adhesion
between the SU-8 template and the elastomer to be cast. Masters were placed in a small desiccator (Fisher Scientific)
along with an open container of (tridecafluoro-1,1,2,2-tetrahydrooctyl) trichlorosilane (Gelest, Inc.)
This desiccator's vacuum port was then connected to a single-stage vacuum pump and evacuated for two hours to produce
a silane coating.

\subsubsection{Elastomer device construction}

Microchannel devices were constructed from polydimethylsiloxane (PDMS, Dow Corning, Sylgard 184). PDMS
elastomer and curing agent were mixed at a ratio of 10:1 by weight, and pored over the photoresist master 
in a plastic petri dish to a depth of about 2 mm.  PDMS was also spun-coat onto a 48 x 60 mm \#1 cover-slip (Gold Seal)
to form the substrate for the device.
Both of these  were then baked at 65\degC for six hours or more to cure
the PDMS.  

Once the PDMS was fully cured, a razor blade was used to carefully cut out a section which encompassed some or all of
the microchannels defined on the photoresist master. This section was then peeled up from the master, revealing
a block of PDMS which contained negative features defining the top and sides of the microchannels.
For each microchannel, three small holes (\~ 0.5 mm) were punched at each entrance using a syringe press, and a larger
hole (\~ 3 mm) was punched at the exit using a biopsy punch.

For each of the ``top side'' PDMS block and the PDMS-coated glass substrate, the PDMS surface was rinsed with 
deionized water and IPA.  Following this, small particles were removed by first laying down and then peeling up
Scotch Magic brand transparent tape.~\ref{rogers-tape-ref}  Each section was then placed below a UV light-emitting
diode with the channel surface facing the diode, and exposed to UV light for ten minutes to promote PDMS-PDMS
adhesion.  After UV exposure, these sections are then firmly pressed together with the channel surface of the top
block against the PDMS-coated surface of the substrate.  The resulting device is then baked at 100\degC for one hour
to promote device bonding.

\subsection{Materials}
The hydrophobic solution was composed of 95 v/o 
tri(methylol propane) triacrylate (TMPTA, Sartomer) and 5 v/o Darocur 1173 photoinitiator (Ciba), 
with 0.005 wt\% methacryloxyethyl thiocarbamoyl rhodamine B (Polysciences) as a cross-linking 
fluorescent dye.
The hydrophilic solution was composed of either 20 mol ethoxylated tri(methylol propane) triacrylate (20-ETMPTA,
Sartomer) or poly(ethylene glycol) diacrylate (PEGDA, $M_n$ = 700, Sigma Aldrich) at 80 v/o, 
15 v/o deionized water, and 5 v/o Darocur 1173 photoinitiator, with 0.005 wt\% 
3,8-dimethacryloyl ethidium bromide (Polysciences) as a cross-linking fluorescent dye.


\subsection{Mask design}

Masks used for single-component fabrication contained two-dimensional arrays of identical aligned 
rods, with a separation in each direction equal to twice the length of the rod to avoid inter-particle
curing (Figure~\ref{fig:rod-masks}(a)). These arrays were designed to maximize the number of 
rods cured per cycle by making them large enough to, at minimum,
cover the field stop aperture for the transmission of the UV beam. This circular aperture 
had a diameter of 1.5 inches.  For example, the photomask containing 500~\microns rod features
was a twenty-by-twenty array, with the 1 mm separation ensuring that the mask area was large enough to
use the full available beam.  
Masks used for Janus fabrication contained only a single line of rod features, with the rods parallel to one 
another and aligned perpendicular to the axis of the line (Figure~\ref{fig:rod-masks}(b)).  Spacing on these 
masks was the same as for single-component fabrication.


\subsection{SFL experiment}

\figone{fig:sfl-experiment-photo}{figures/rods/microchannel-experiment.JPG}{0.8\linewidth}{
A microfluidic device platform containing multiple Y-junction microchannels is placed on 
the microscope used for SFL, and connected to two pressure sources to pump PEGDA and TMPTA 
monomer solutions.}

\figone{fig:sfl-schematic}{figures/literature-review/doyle-sfl-schematic.png}{0.8\linewidth}{
General schematic of SFL experiment, illustrating computer-controlled operation of the pressure source,
three-way valve and shutter to control flow and exposure conditions. Figure from~\cite{dendukuri-sfl}}

\figone{fig:janus-sfl-schematic}{figures/rods/janus-rod-sfl-schematic.png}{0.70\linewidth}{
SFL fabrication of Janus rods.}

UV exposure and experimental imaging for small-rod fabrication was carried out using a 60x 
oil-immersion objective lens (Olympus America), with an additional 1.6x lens added to the beam path for
the fabrication of smaller rods.  Using the 60x objective, a demagnification factor of approximately 
33 was typically observed between the mask and the resulting rods; i.e., a rod of 500~\microns length defined on
the photomask would typically result in the fabrication of a 12~\microns rod in the microchannel.

Microchannel flow was driven by gas pressure supplied by a house nitrogen line or a compressed air tank 
(SJ Smith Welding Supply). Pressure control was achieved using a custom-built pressure box, consisting of
four computer-controlled regulators and four duplex valves connected to a USB
controller (National Instruments) allowing for 
up to four independent pressures driving up to eight separate lines. UV light was supplied by a ?? W mercury lamp
connected in fluorescence microscope configuration, with exposure time controlled using a Lambda SC 
electronic shutter (Sutter Instruments).

The SFL experiment was driven using LabView software (National Instruments) with a custom
interface (Figure~\ref{fig:sfl-labview}). The process
flow of an SFL experiment is illustrated in Figure~\ref{fig:sfl-flowchart}, and consists of the following steps:

\begin{enumerate}
\item Specify ``flow time'' $t_{flow}$, ``pause time'' $t_{pause}$, and exposure time $t_{expose}$.
\item Set pressure for each port on electronic regulators.
\item Until the experiment is stopped:
\begin{enumerate}
\item Open valves. Wait $t_{flow}$.
\item Close valves. Wait $t_{pause}$.
\item Open shutter. Wait $t_{expose}$.
\item Close shutter. Repeat.
\end{enumerate}
\end{enumerate}

For a schematic of the SFL experimental setup, see Figure~\ref{fig:sfl-schematic}.
For each monomer solution, a 10-100\uL pipette tip (Eppendorf) was filled with approximately 60\uL of solution.
A silicone tube was connected at one end to the output of one valve of the pressure box, and the other end
was inserted into the top of the pipette tip.  The bottom of the pipette tip was then inserted into one of the
small entrance holes in the microchannel device.  This device was the placed on above the microscope objective,
and the microscope was focused such that the focal place was inside the channel.  After starting the 
experiment, the shutter was allowed to open and the fine focus was adjusted manually to optimize
curing conditions. Optimization was judged by visual inspection of the resulting particles. 

Single-component rods were fabricated using the central entrance channel to supply monomer to the system, 
leaving the other two entrances unused. Typical experimental settings for these experiments were pressure
8 psi, flow time 1.5 s, and pause time 2 s.  Exposure times for PEGDA or 20-ETMPTA were typically 0.05-0.15 s
depending on the mask dimensions, while exposure times for TMPTA were typically 0.2-0.3 s to account for
the less favorable curing behavior of the hydrophobic monomer.  The fabrication of smaller rods generally
necessitated slightly higher exposure times.  

Janus rods were fabricated using the two side entrance channels to supply each type of monomer solution,
with the central channel left unused.  (All three channels were used for fabricating branched Janus particles; see
section~\ref{sec:SFLx3}.)  Typical experimental settings for these experiments were pressure 7 psi on each side, 
flow time 2 s, pause time 2 s, and exposure time 0.2-0.3 s depending on the mask dimensions.
The two monomer streams would enter the main channel on either side, and form
an interface along the center of the channel (Figure~\ref{fig:Janus-channel}). Because the hydrophobic and
hydrophilic streams were immiscible and the microchannel flow took place in the laminar flow regime~\ref{?}, 
this interface persisted for several hundred microns down the channel.
During fabrication, the curing regions would be aligned such that each rod straddled this interface,
producing rods which incorporated both materials.

In either case, the resulting particles were ejected at the end of the experiment into the final reservoir, which
was left empty to allow collection of the maximum volume of particles possible.

\subsection{Particle collection and purification}

Particle collection techniques were developed and optimized for a number of different experimental cases,
depending both on the particular monomer solutions involved and on the desired final rod behavior, i.e. 
free diffusion or the self-assembly of the hydrophobic or hydrophilic material.  Details of the differences
in collection techniques and their effectiveness will be explored in Section~\ref{sec:rod-collection-results}.

In general, SFL-fabricated rods were ejected, still in monomer solution, from the fabrication microchannel into 
an open reservoir, with the bottom and sides formed of PDMS and the top open to the environment.  Owing to the
surface tension of the monomer, these particles generally stayed near the side walls of the reservoir, often
collecting in corners or imperfections of the wall.  Note that in the single-component case, the particles in
the reservoir may be assumed to be perfectly non-interacting as they are still suspended in the monomer solution, 
a solvent with the
same composition as the particles.   In the case of Janus rods this assumption is not necessarily valid, as
the particles are suspended in a mixture of the two different monomer solutions.  Fabrication experiments were
generally stopped well before the monomer solutions could fill the reservoir; an hour-long SFL fabrication experiment
may be expected to use a total volume of monomer solution equivalent to only 1/10th the volume of the
reservoir.  Once fabrication 
was complete, the collection procedure generally consisted of the following steps:

\begin{enumerate}
\item A \textit{collection solvent} was introduced into the reservoir via the top opening, with volume sufficient to
fill the reservoir (generally $\sim$ 20\uL).
\item A slight delay was allowed for the particles to disperse into the new solvent from their 
original position along the 
walls. This delay varied depending on the solvent viscosity, but was generally between 30 and 120 s.
\item A pipette was positioned in the reservoir with the tip near the highest concentration of particles, estimated
by eye using the microscope.
\item The pipette was used to collect a volume equivalent to the amount of collection solvent introduced.
\item Repeat 3-5 times.
\end{enumerate}

The particular solvent used, the dispersal time, and the type of pipette were all varied for different experimental
cases.

Following pipette collection, the particles were either transferred into a micro-centrifuge tube for solvent 
exchange, or directly into an observation chamber.  Solvent exchange was carried out in cases where the desired
observation solvent was either incompatible with PDMS, such as toluene, or of sufficiently high viscosity to make
direct collection in this solvent difficult, such as collecting in PEGDA.  Solvent exchange was also used to reduce
the concentration of the monomer solution components in the final solution, i.e. to remove liquid monomer and 
reduce the background fluorescence due to free dye.

To carry out solvent exchange, particles in a micro-centrifuge tube were either centrifuged at 3000 rpm for 10 minutes or
allowed to settle due to gravity for 6+ hours in order separate particles from solution.  A quantity of solution equivalent
to one-half the total volume was then carefully pipetted from the top of the tube, and replaced with the same quantity of 
the desired \textit{observation solvent}.  The tube was then agitated using a vortex mixer in order to mix the solvents and
re-suspend the particles.  This process was planned to be repeated seven times in order to increase the volume fraction of
the new observation solvent to over 99\%.  However, experimental results indicating that this procedure was resulting in heavy
particle losses resulted in some experiments only involving three or four rinse cycles.  These results will be detailed in 
Section~\ref{sec:rod-collection-techniques}.

\mnote{Add better details on glass observation chambers. Also explain constraints on observation chambers.}

\figone{fig:confocal-chamber-cartoon}{figures/rods/confocal-chamber-cartoon.png}{0.50\linewidth}{
Design of a glass observation chamber for microscopy of particles.}

Observation chambers were constructed by affixing a glass coverslip to one end of short glass tube using 
five-minute epoxy (3M).  Oriented with the coverslip as the base, this provides a flat transparent surface
for microscopy. (See Figure~\ref{fig:confocal-chamber-cartoon}


\subsection{Diffusion Measurements}
\label{sec:exp-diffusion}

To carry out diffusion measurements, the rod sample was placed in the solvent of interest and pipetted into an observation 
chamber.  In some cases the observation chamber was pre-coated with no-stick silane ((tridecafluoro-1,1,2,2-tetrahydrooctyl) 
trichlorosilane, Gelest, Inc.) to inhibit sticking between the rods and the chamber surfaces; in others,
the observation chamber was partially pre-filled with solvent in order to pre-treat the 
bottom surface, and the rods were allowed to settle for 20-30 minutes before observation.

The observation chamber was then placed on the sample stage of a fast confocal laser scanning microscope (CLSM; 
an IX71 inverted microscope, Olympus, Inc. connected to a vt$^{eye}$ confocal scanner, Visitech International). The CLSM
included two independent excitation laser lines with peak wavelengths of 491 nm and 561 nm, corresponding to our ethidium 
bromide and Rhodamine B dyes.  A 100x oil-immersion objective lens was used for imaging the rods which had settled at the bottom
of the chamber.  The microscope was initially used in bright-field imaging mode and manually adjusted to find the particles and 
optimize the focus.

Following the bright-field adjustments, the microscope was put into confocal mode, the laser line selected 
based on the particular sample, and two-dimensional rod diffusion at the 
bottom of the chamber was imaged in time-series mode.  The microscope was configured such that images were acquired at
a rate of 1 frame per second (fps), consisting of the average of eight frames acquired at 30 fps.  
(For fast diffusion of the smallest rods,
image acquisition was adjusted to 2 fps.)  Laser exposure was triggered only during active 
acquisition, and shuttered otherwise.  Image spatial resolution was 512 $\times$ 512 pixels with a scale of 10 pixels
per \microns, and each pixel's intensity was scaled over eight bits to assume values in the range of 0--255.
The intensity scale was adjusted relative to the actual intensity to maximize the dynamic range for the particular sample.
The total time of a typical experiment was 10 minutes.

Time-series image data were saved as 8-bit TIFF image stacks and transferred to group computation servers for analysis.
Analyses were carried out to compute mean-square-displacement and orientation displacement for each rod using algorithms as 
described in Section~\ref{sec:rod-tracking} and implemented in Matlab scripts described in Section~\ref{sec:rods-implementation}, 
and these data
were used to calculate two-dimensional diffusion constants.  

\subsection{Tiled microscopy measurements}
\label{sec:tiled-microscopy}

%%%%%%%%%%%%%%%%%%%%%%%%%%%%%%%%%%%%%%%%%%%%%%%%%%%%%%%%%%%%%%%%%%%%%%%%%
\section{Results and Discussion}

\subsection{Resolution}

%%%%%%%%%%%%%%%%%%%%%%%%%%%%%%%%%%%%%%%%%%%%%%%%%%%%%%%%%%%%%%%%%%%%%%%%%
\subsection{Collection techniques}
\label{sec:rod-collection-techniques}

\subsubsection{Collection solvents}

\subsubsection{Pipettes}

\subsubsection{Solvent exchange}

\subsection{Surface interactions}
\label{sec:surface-interact}

%%%%%%%%%%%%%%%%%%%%%%%%%%%%%%%%%%%%%%%%%%%%%%%%%%%%%%%%%%%%%%%%%%%%%%%%%
\subsection{Translational and Rotational Diffusion}

\figfour{fig:rods-msd}{figures/rods/dynamics-data/msd6um.png}{figures/rods/dynamics-data/msd9um.png}{figures/rods/dynamics-data/msd12um.png}{figures/rods/dynamics-data/msd15um.png}{Mean-square displacement plots for (a) 6 \microns~rods, (b) 9 \microns~rods, (c) 12 \microns~rods and 15 \microns~rods.}

\figfour{fig:rods-rot}{figures/rods/dynamics-data/rot6um.png}{figures/rods/dynamics-data/rot9um.png}{figures/rods/dynamics-data/rot12um.png}{figures/rods/dynamics-data/rot15um.png}{Mean-square rotational displacement data plots for (a) 6\microns~rods, (b) 9 \microns~rods, (c) 12 \microns~rods and 15 \microns~rods.}

\subsubsection{Experiments}

Dynamics experiments were carried out to study the effect of rod size and aspect ratio on the rate of
diffusion.  To maximize rod mobility in this study, the results of Section~\ref{sec:surface-interact}
were taken into account and single-component TMPTA 
rods with lengths of 6, 9, 12 and 15 \microns~were suspended in toluene and transferred 
to glass-bottom confocal observation chambers which had 
also been
pre-rinsed in toluene.  These chambers were observed by CLSM under the conditions described in 
Section~\ref{sec:exp-diffusion} for time-series experiments.  The resulting 2D image series were 
processed and analyzed using the algorithm described in Section~\ref{sec:rod-tracking}, and 
the translational and rotational mean-square displacements (MSD) were calculated and plotted.
The translational MSD are plotted with respect to time in Figure~\ref{fig:rods-msd}, and the 
rotational MSD are plotted with respect to time in Figure~\ref{fig:rods-rot}.



\subsubsection{Compare to theory}

%%%%%%%%%%%%%%%%%%%%%%%%%%%%%%%%%%%%%%%%%%%%%%%%%%%%%%%%%%%%%%%%%%%%%%%%%%%
\subsection{Self-Assembly of Janus Rods}

\figone{fig:big-janus-high-conc}{figures/rods/large-janus-high-conc.png}{0.8\linewidth}{
Large Janus rods (45 x 11 x 11~\microns), randomly oriented.}

\figtwo{fig:big-janus-assembly}{figures/rods/large-janus-rods-45x11x11um.jpg}{figures/rods/large-janus-self-assembly.jpg}{
Large Janus rods (45 x 11 x 11~\microns) self-assemble following agitation.}

Initial hydrophobic/hydrophilic Janus fabrication experiments 
were performed using microchannels fabricated with a height of 15 \microns, 
producing relatively large Janus particles (Figure~\ref{fig:big-janus-high-conc}). These particles had dimensions 
of roughly 45 \by 11 \by 11 \microns, and exhibited little spontaneous self-assembly in water.  However, agitation
of these particles to drive their rearrangement into new configurations produced substantial substantial
self-assembly (Figure~\ref{fig:big-janus-assembly}), producing clustered structures with high contact
areas between the hydrophobic regions of the particles.  This served as a proof-of-concept of 
hydrophobic self-assembly and duplication of the
results from Dendukuri \textit{et al.}~\cite{dendukuri-amph}.  

\figthree{fig:janus-sizes}{figures/rods/janus-rods-4x2um.png}{figures/rods/janus-rods-6x2um.png}{figures/rods/janus-rod-cluster-zoom.png}{Various sizes of Janus rods.}

Next, the fabrication microchannel was reduced in height to 7 \microns~and the mask size was varied to explore the
size limitations in fabricating Janus rods.  ...

\figone{fig:assembly-v-solvent}{figures/rods/janus-rod-assembly-vs-solvent.png}{0.8\linewidth}{
Self-assembly of Janus rods in (a) water, (b) DMSO and (c) IPA.}

To explore the effect of assembly solvent on the resulting structures, identical Janus rods with dimensions 
12 \by 3 \by 3 \microns~were fabricated and placed in three different solvents: water, dimethyl sulfoxide 
(DMSO) and isopropyl alcohol (IPA).  In water, we see a result similar to that seen for the large Janus rods,
with strong hydrophobic self-assembly producing structured clusters, with strongly-correlated orientations
between the rods within a cluster.  Suspension and agitation in DMSO, however, shows little or no self-assembly.
This result is consistent with DMSO's status as a less polar solvent than water, decreasing the driving force
for hydrophobic self-assembly.  Moving the particles to IPA, a non-polar solvent, shows a completely different
self-assembly behavior following agitation, in which the green-dyed hydrophilic parts of the particles are 
assembling to maximize contact area.  This series of experiments illustrates the potential for 
control of self-assembly via varying the solvent conditions.

\figone{fig:assembly-small-clusters}{figures/rods/janus-rods-large-clusters-ordered.png}{0.8\linewidth}{
Janus rods self-assemble at low concentrations into small ``micellar'' structures.}

\figone{fig:large-struct}{figures/rods/janus-rods-large-structures.png}{0.8\linewidth}{
Janus rods self-assemble at high concentration into extended, gel-like structures in water.}

Focusing on self-assembly in water, we see that a variety of different structures may be 
achieved.  In a single sample, local variations in concentration due to uneven mixing may produce
dramatically different results.  Where the local concentration is low, highly-ordered clusters 
are produced which have a ``micelle-like'' geometry, with the particles grouped around a common 
center where the hydrophobic regions tend to maximize contact area 
(Figure~\ref{fig:assembly-small-clusters}).  Where the local concentration
is high, however, extended structures containing many more particles may be formed
(Figure~\ref{fig:large-struct}).  These
structures have a ``gel-like'' motif in that they are more randomly arranged, but still contain
sub-clusters which are formed by hydrophobic assembly.

\figtwotb{fig:large-area}{figures/rods/janus-tiled-large-area.jpg}{figures/rods/high-conc-large-area.png}{
A large-area view of Janus rod self-assembly in DMSO/water at a low concentration.}


\figone{fig:bonds-per-particle}{figures/rods/struct-plot-2.jpg}{0.8\linewidth}{
Bonds per Janus particle.}

\figone{fig:orientation-plot}{figures/rods/struct-plot-1.jpg}{0.8\linewidth}{
Orientational ordering.}

A more quantitative study of Janus rod self-assembly was attempted by fabricating large 
numbers of Janus rods and observing their self-assembly in confocal observation chambers.
Fabrication sample size for each experiment was approximately 50,000 particles; losses
due to transfer and processing reduced this to an estimated 20,000 particles.
Two parameters were varied: the assembly solvent, which was a mixture of DMSO and water
with the relative fraction adjusted; and the aspect ratio of the particle, which was varied
in the range of 1.5-4.5.  Large-area images were acquired as detailed in 
Section~\ref{sec:tiled-microscopy}, and analyzed as detailed in Sections~\ref{sec:rod-tracking}
and~\ref{sec:structure-calcs}.  Sample images at low and high magnification are shown
in Figure~\ref{fig:large-area}.

Two parameters were calculated for each sample: the average number of contacts or ``bonds''
per particle, and the degree of orientational ordering for particles in the same cluster.
The number of ``bonds'' was calculated as the number of particles which were found within a 
center-to-center distance equal to one rod length.  The orientational ordering was measured
as the average dot-product of the orientation vectors between pairs of rods which were 
considered to be in the same cluster, with clusters defined according to the same bond criterion.
The results of these calculations are plotted for all experiments in Figures~\ref{fig:bonds-per-particle}
and~\ref{fig:orientation-plot}.  Unfortunately these results are somewhat contradictory, and it 
is difficult to draw good conclusions about the behavior of Janus rods undergoing self-assembly;
but it is possible to make a few general comments.

The first thing that should be noted is that for all experiments, the average number of bonds per
particle never exceeds about 0.6, and in fact most experiments have bond numbers below 0.2.  This
can be attributed to the fact that despite the high number of particles fabricated relative to
typical SFL experiments, the rod concentrations are still quite low: settled to the bottom of the
container, the average areal concentration never exceeds about 5\%.  The bond numbers for 
DMSO/water mixtures of 50--100\% water are all relatively similar, in the range of 0.1--0.2, and do
not vary much with aspect ratio.  

For solutions with greater than 
50\% DMSO, however, we see a striking increase in bond number.  Without carrying out additional experiments it 
is difficult to determine the cause of this increase. We might speculate that the higher density of DMSO
and the consequent slower settling might allow the rods more time to ``find'' one another and assemble, but in
this case we would expect a more continuous variation of bond number with DMSO concentration. Another cause might
be some experimental error which increased the final particle yield in the high-DMSO samples relative to the others.
Within these samples, however, we see a consistent variation in bond number with respect to aspect ratio, with
much higher bond numbers observed at low and high aspect ratios.  The increase in bond number at low aspect ratio, 
and thus low particle size, might be attributed to the higher mobility of these particles on a surface where
Janus particles would tend to stick (see Section~\ref{sec:surface-interact}), while the increase at high aspect
ratio may be due to the higher opportunity for contact available with longer rods.

Turning our attention to the orientational ordering, we see much more consistent results across the
range of solvent compositions, with little in the way of consistent discernible variation between samples with
the same aspect ratio.  A dot-product of 1.0 implies perfect parallel ordering, while a dot product of 0.0 implies
perpendicular ordering; average values close to 0.5 may be interpreted as showing relatively random orientation
within the sample.  With that in mind, it is interesting to note that particles with low aspect ratio show 
a relatively strong ordering, and that the dot product decreases towards randomness as the aspect ratio 
increases.  This is somewhat counter-intuitive, and contradicts some previous imaging which would suggest
better correlation between particles with a high aspect ratio.

%%%%%%%%%%%%%%%%%%%%%%%%%%%%%%%%%%%%%%%%%%%%%%%%%%%%%%%%%%%%%%%%%%%%%%%%%%%%%%%%%%%%
\subsection{Potential applications}

\figtwotb{fig:silver-janus}{figures/rods/silver-microchannel-twostream.png}{figures/rods/silver-fluorescence-assembly.png}{
(a) Ag nanoparticles are suspended in PEGDA monomer solution for Janus SFL fabrication, and
(b) Ag nanoparticles are embedded in at 5 wt\% in otherwise-standard PEGDA monomer solution, and Janus rods
are fabricated as a demonstration of self-assembled functional Janus rods.}

The potential for the incorporation of functional materials into SFL Janus particles was 
explored in a pilot experiment involving the incorporation of Ag nanoparticles (NPs) into the
hydrophilic PEGDA monomer solution at 5 wt\%. A Janus fabrication experiment using this 
solution was set up, and the Ag NPs were shown not to interfere with the establishment of a stable
pair of co-flowing streams (see Figure~\ref{fig:silver-janus}(a)). Standard Janus SFL was carried
out using a 400 \microns~mask to produce 12 \microns~Janus rods. When placed in water, these rods
self-assembled normally, as shown in Figure~\ref{fig:silver-janus}(b).  Note, however, the presence of
free Ag NPs (fluorescing red) both in the free solution and aggregated around the assembled cluster.
This illustrates a major difficulty in incorporating any functional material in SFL, i.e. the successful 
rinsing and removal of functional materials from the suspension solvent.

%%%%%%%%%%%%%%%%%%%%%%%%%%%%%%%%%%%%%%%%%%%%%%%%%%%%%%%%%%%%%%%%%%%%%%%%%%%%%%%%%%%%
\subsection{Potential connections to theory}
\figone{fig:assembly-sim}{figures/rods/schweizer-tripathy-janus-rod-simulation.png}{0.8\linewidth}{
Simulation of Janus rod assembly.}



\chapter{Fabrication and Self-Assembly of Exotic Colloids}
\label{ch:exotic}

\section{Introduction}

Stop-flow lithography (SFL) has been demonstrated to be useful for fabricating particles with 
simple geometric and chemical anisotropies such as a high aspect ratio and 
two-sided Janus functionality.
With this shown, the next question becomes, can the same technique be used to produce other forms of 
anisotropy?  In this chapter, the fabrication of particles with branched anisotropy is explored, and 
the hydrophobic self-assembly of these particles is demonstrated in polar solvent.  More exotic 
functionalities are also explored, including particles whose assembly may take different forms 
depending on the driving force.


\section{Experimental Procedure}
\label{sec:SFLx3}

\subsection{Three-stream SFL experiment}

\figone{fig:three-stream}{figures/complex-shapes/three-stream-fab.jpg}{0.6\linewidth}{
Fluorescence image of a microchannel during three-stream flow.}

SFL experiments were carried out using the same device design as in Section~\ref{sec:sfl-expt-rods} 
(see Figure~\ref{fig:device-design}).  All three input channels were used for the fabrication of 
branched and other exotic particles, with the typical monomer inputs being PEGDA solution in
the center channel and TMPTA solution on the left- and right-hand inputs. (See 
Section~\ref{sec:janus-materials} for a full description of these solutions.) 
Typical initial pressures for microchannel
flow were 8 $psi$ for the left- and right-hand inputs and 6 $psi$ for
the center input. Typical values for $t_{flow}$, $t_{pause}$ and $t_{expose}$ were 2.0 s, 2.0 s and 0.25 s, 
respectively.  SFL fabrication masks were designed for three-stream fabrication in a manner similar to 
the masks used for the fabrication of Janus rods, with several identical particle patterns arranged
in a single line, which could then be aligned to allow polymerization overlapping the 
two parallel TMPTA/PEGDA interfaces.

All SFL parameters were adjusted at the beginning of each experiment to optimize fabrication 
conditions.  A number of constraints were applied to achieve good results for 
the fabrication of multiple-component particles, including:

\begin{itemize}
\item $t_{flow}$ sufficient to eject particles following polymerization.
\item $t_{pause}$ sufficient for a complete stop in flow, to optimize polymerization conditions.
\item $t_{pause}$ small enough to prevent the center stream from breaking up into droplets.
\item Relative pressures optimized to produce a jetting flow with a sufficiently narrow stream
for fabricating small particles.
\item $t_{expose}$ sufficient to fully cure particles.
\end{itemize}

The major difficulty in this experiment was balancing the relative pressures and flow times to produce
PEGDA/TMPTA interfaces which were sufficiently narrow for the fabrication of the particles of interest,
while still avoiding the break-up of the center stream into a series of droplets (``dripping mode'').
This set of constraints imposed strict limits on the minimum size of the particles, with
the minimum width achieved for the center stream being approximately 12 \microns.
Due to the presence of air bubbles and imperfections in the microchannel, this optimization had to be
performed for every experiment, and greatly reduced consistency between subsequent fabrication experiments.

\subsection{Collection and processing}

Complex Janus particles were ejected from the fabrication channel, still in monomer solution, and collected
in an open reservoir.  The particle collection procedure was identical to that outlined in 
Section~\ref{sec:exp-collection}; ethanol was used as the collection solvent, and glass Pasteur
pipettes were used to transfer the particles to extract the particles.  All particles samples
were washed by microcentrifuge solvent exchange using ethanol, and were transferred to 
glass observation
chambers either in ethanol (for simple fluorescence observation) or in water (for
observation of self-assembly).


\section{Results and Discussion}

\subsection{Branched particles}

\figone{fig:branching-anisotropy}{figures/complex-shapes/branching-particles-anisotropy-dimension.png}{0.6\linewidth}{
Glotzer anisotropy dimension: branching.~\cite{glotzer-solomon}}

SFL fabrication experiments were performed with masks designed to target the branching morphologies
portrayed in the scheme of anisotropy dimensions outlined by Glotzer and Solomon
(see Figure~\ref{fig:branching-anisotropy}).~\cite{glotzer-solomon}  As single-component rods and 
two-part Janus rods had been accomplished in Chapter~\ref{ch:rods}, the following particle 
morphologies were targeted: hydrophilic rods with hydrophobic patches on either end; three-branch particles
with hydrophobic ends; and four-branch particles with hydrophobic ends.


\begin{figure}[h]
\begin{center}
\includegraphics[height=1.5in]{figures/complex-shapes/twopatch-3chain-sb20.png} \includegraphics[height=1.5in]{figures/complex-shapes/twopatch-flchain-sb20.png}
\end{center}
\caption{Chain-like assembly of tall two-patch particles. Scale-bar is 20 \microns.}
\label{fig:tall-2patch}
\end{figure}

\begin{figure}[h]
\begin{center}
\includegraphics[height=1.5in]{figures/complex-shapes/4patch-sb20-01.png} \includegraphics[height=1.5in]{figures/complex-shapes/4patch-sb20-02.png} \includegraphics[height=1.5in]{figures/complex-shapes/4patch-sb20-03.png}
\end{center}
\caption{Self-assembly of tall four-patch particles. Scale-bar is 20 \microns.}
\label{fig:tall-4patch}
\end{figure}

As with Janus rods, initial experiments were carried out in microchannel devices with a height of
15 \microns~to produce larger particles which might be more easily studied first.  These proof-of concept
experiments were carried out to produce two-patch and four-patch particles with heights of roughly 11
\microns~and width of 20 \microns.  

These particles were then transferred into water suspension to 
demonstrate their self-assembly.  Two-patch particles were observed to form linear chain-like structures
with end-to-end assembly, as seen in Figure~\ref{fig:tall-2patch}.  This is dramatically different from
the micelle-like clusters formed by Janus rods, shown previously in 
Figure~\ref{fig:assembly-small-clusters}.  Four-patch particles assembled into denser clusters due
to their additional assembly sites, producing structures which combined chain-like 
morphologies (Figure~\ref{fig:tall-4patch}(c)) with dense and loop-like structures 
(Figure~\ref{fig:tall-4patch}(a,b)).

\begin{figure}[h]
\begin{center}
\includegraphics[height=1.5in]{figures/complex-shapes/two-patch-rods-single-image.png} \includegraphics[height=1.5in]{figures/complex-shapes/janus-threepatch-single-image.png} \includegraphics[height=1.5in]{figures/complex-shapes/janus-crosses-single-image.png}
\end{center}
\caption{Two-component particles with (a) two, (b) three and (c) four hydrophobic patches.}
\label{fig:branched-series}
\end{figure}

\begin{figure}[h]
\begin{center}
\includegraphics[width=0.6\linewidth]{figures/complex-shapes/crosses-high-conc.png}
\end{center}
\caption{Janus crosses at high concentration.}
\label{fig:crosses-high-conc}
\end{figure}

Following this demonstration, subsequent experiments focused on producing high-quality 
branched Janus particles using microchannels with a height of 7 \microns.  These experiments were carried
out to produce good examples of particles with two, three and four patches, with heights of
3 \microns~and lateral dimensions of approximately 30 \microns.  

Unfortunately, due to the considerations
outlined in Section~\ref{sec:SFLx3}, these experiments failed quickly and rarely produced more than a few
particles at a time.  One large sample of four-patch particles was produced, and a fluorescence 
microscopy image of this sample is shown in Figure~\ref{fig:crosses-high-conc}.  Here we can see 
some strongly-aligned self-assembly, with crosses generally arranging themselves such that
their hydrophobic ends are maximizing contact area.  This is a flexible mode of self-assembly 
which may enable the formation of many different superstructures, and this image alone shows several different
cluster morphologies.

\subsection{Multiple modes of assembly}

\begin{figure}[h]
\begin{center}
\includegraphics[width=0.4\linewidth]{figures/complex-shapes/boomerangs-dense-assembly.jpg} \hfill \includegraphics[width=0.4\linewidth]{figures/complex-shapes/boomerangs-open-assembly.jpg}
\end{center}
\caption{Two possible modes of self-assembly for Janus ``boomerangs'': (a) geometric, and (b) open and hydrophobic}
\label{fig:boomerang-assembly}
\end{figure}

\figone{fig:boomerangs}{figures/complex-shapes/janus-boomerangs-single-image.jpg}{0.7\linewidth}{
Fluorescence image of fabricated Janus ``boomerangs''.}

In addition to the branched particles illustrated above, we attempted to design and fabricate 
a particle which might undergo two or more different types of self-assembly depending on the environment
in which it was placed.  

To achieve this, we designed a ``boomerang'' particle incorporating 
hydrophobic ends.  Under neutral solvent conditions, as in a weakly-polar solvent such as DMSO, these particles
might be induced to undergo a 
dense geometric self-assembly with cluster morphology dictated by the 
particle shape; an example of such a structure is illustrated in Figure~\ref{fig:boomerang-assembly}(a).
Such an assembly might be induced by the introduction of a low-molecular-weight polymer to indue a 
depletion interaction.  To achieve the second assembly mode, the particles would be 
placed in a strongly-polar solvent such as water in order to drive a hydrophobic attraction between 
the particle ends, producing a more randomly-oriented and open structure such as the one illustrated
in Figure~\ref{fig:boomerang-assembly}(b).

Large boomerang-type particles were successfully fabricated by SFL and imaged via fluorescence 
microscopy (see Figure~\ref{fig:boomerangs}).  However, while these particles were successfully 
observed in DMSO, we failed to induce either a depletion interaction or a strong hydrophobic 
self-assembly.


\chapter{Conclusions}
\label{ch:conclusions}

In this work we studied the fabrication, self-assembly and dynamical behavior
of shape- and chemically-anisotropic colloidal particles.  The fabrication 
of colloidal rods of various sizes and aspect ratios was demonstrated via
stop-flow lithography for rods with
hydrophilic, hydrophobic and Janus functionalities.  Image processing algorithms
were developed for identifying, locating and measuring the orientation of 
colloidal rods in microscopy images, and software was implemented to carry out
this analysis and characterize structural and dynamical properties.
Single-component particle 
dynamics were observed via time-series confocal microscopy using particles of 
various aspect ratios to study the effects of particle geometry on diffusion.
The self-assembly of Janus rods was studied using
3D confocal microscopy for particles of various aspect ratios,
in a number of different solvent conditions, to determine the effects of size 
and attraction strength on the resulting structure.  The fabrication of more complex
multiple-patch particles with several different geometries was also demonstrated via SFL.

\section{Future work}

Significant challenges were identified in the areas of particle collection and processing as 
detailed in Section~\ref{sec:rod-collection} which may have significantly affected the 
results of dynamics and self-assembly experiments.  It might be possible to solve these
processing issues using some non-stick coating instead of the fluorosilane coating 
which was used.  Another avenue to explore includes the use of a microfluidic concentration
and cleaning solution to exchange the monomer solution for another solvent, 
which may be integrated with the fabrication system.  

Once these issues have been solved, a number of interesting studies on the dynamics and
structure of suspensions of microfabricated colloids are possible.  This includes studying 
self-assembly and dynamics of colloidal suspensions at high concentrations, as well as 
the behavior of biphasic suspensions containing a mix of particles with different geometries and
functionalities.

It should also be possible to develop image processing algorithms to track particles with more
complex geometries.  The algorithm developed in Chapter~\ref{ch:comp-tracking} is independent 
of particle geometry throughout the image cleanup, segmentaton and skeletonization steps.
The development of algorithms which could translate the skeletons of other particle 
geometries into position and orientation information would greatly expand the 
capabilities of particle tracking.


\appendix

\chapter{Microfluidic Devices for Studying Self-Assembly}
\section{Introduction}

\begin{itemize}
\statement{SFL is low-scale technique for fabricating very small particles}
\statement{Most interesting physics occurs at higher concentrations}
\statement{SFL yields are low once particles are transferred out of device}
\statement{Single microfluidic system to fabricate and study particles is desirable}
\end{itemize}

\section{Experimental Procedure}
\subsection{Device Design and Fabrication}

\tempfigure{Overall design}
\tempfigure{Post filters; Channel-height filters}
\begin{itemize}
\done{Design capable of multi-stream fabrication}
\done{Design capable of concentrating particles in a small container}
\done{Multi-layer SU-8 master fabrication}
\done{Initial filter design: posts}
\done{Final filter design: channel height}
\done{Concentrator geometries}
\done{System to exchange solvents and clean particles}
\end{itemize}

\subsection{Proposed Protocol}

\tempfigure{Cartoon of proposed protocol}
\begin{itemize}
\done{Fabricate particles in channel}
\done{Particles collect in concentration chamber}
\done{When finished, cure fab channel shut}
\done{Rinse particles}
\done{Agitate to break up structure}
\done{Image}
\end{itemize}


\section{Results and Discussion}

\subsection{Concentrator results}

\tempfigure{Janus particles in concentrator}
\tempfigure{Failed devices}
\begin{itemize}
\done{Particle fabrication: slowed by pressure}
\done{Particle concentration by filters}
\done{Solvent exchange: challenges due to pressure buildup}
\done{Aggregates refuse to break up: suggested explanations}
\done{Device failures}
\end{itemize}

\section{Future directions}

\begin{itemize}
\statement{Suggestions on device design}
\statement{Suggestions for other uses of these devices}

\end{itemize}


\chapter{Grayscale Stop-Flow Lithography}
\section{Introduction}

\begin{itemize}
\statement{SFL particle height set by channels; low versatility}
\statement{Allow height variation within channel?}
\statement{Give examples of grayscale lithography}
\end{itemize}

\section{Experimental Procedure}
\subsection{Design of Grayscale Filters}

\tempfigure{Plot transmission vs filter thickness}
\begin{itemize}
\done{Construct mixed-PDMS filters}
\end{itemize}

\subsection{Grayscale Stop-Flow Lithography}

\tempfigure{Cartoon of grayscale SFL fabrication}
\begin{itemize}
\done{Placement in SFL beam path}
\done{Experimental optimization}
\end{itemize}

\section{Results and Discussion}

\subsection{Resulting particles}
\begin{itemize}
\done{Achieved height variation}
\done{Particle swelling issues observed}
\end{itemize}

\section{Future work}
\begin{itemize}
\notdone{True grayscale masks}
\end{itemize}

\end{document}
